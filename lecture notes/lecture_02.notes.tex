 %!TEX TS-program = xelatex
%!TEX encoding = UTF-8 Unicode

%\def \papersize {a5paper}
\def \papersize {a4paper}
%\def \papersize {letterpaper}

%\documentclass[14pt,\papersize]{extarticle}
\documentclass[12pt,\papersize]{extarticle}
% extarticle is like article but can handle 8pt, 9pt, 10pt, 11pt, 12pt, 14pt, 17pt, and 20pt text

\def \ititle {Origins of Mind: Lecture Notes}
\def \isubtitle {Lecture 01}
%comment some of the following out depending on whether anonymous
\def \iauthor {Stephen A.\ Butterfill}
\def \iemail{s.butterfill@warwick.ac.uk% \& corrado.sinigaglia@unimi.it
}
%\def \iauthor {}
%\def \iemail{}
%\date{}

%\input{$HOME/Documents/submissions/preamble_steve_paper4}
\input{$HOME/Documents/submissions/preamble_steve_lecture_notes}

%no indent, space between paragraphs
\usepackage{parskip}

%comment these out if not anonymous:
%\author{}
%\date{}

%for e reader version: small margins
% (remove all for paper!)
%\geometry{headsep=2em} %keep running header away from text
%\geometry{footskip=1.5cm} %keep page numbers away from text
%\geometry{top=1cm} %increase to 3.5 if use header
%\geometry{bottom=2cm} %increase to 3.5 if use header
%\geometry{left=1cm} %increase to 3.5 if use header
%\geometry{right=1cm} %increase to 3.5 if use header

% disables chapter, section and subsection numbering
\setcounter{secnumdepth}{-1}

%avoid overhang
\tolerance=5000

%\setromanfont[Mapping=tex-text]{Sabon LT Std}


%for putting citations into main text (for reading):
% use bibentry command
% nb this doesn’t work with mynewapa style; use apalike for \bibliographystyle
% nb2: use \nobibliography to introduce the readings
\usepackage{bibentry}

%screws up word count for some reason:
%\bibliographystyle{$HOME/Documents/submissions/mynewapa}
\bibliographystyle{apalike}


\begin{document}



\setlength\footnotesep{1em}






%---------------
%--- start paste


      
I seize little Isabel and swing her around, thereby making her laugh and 
breaking a vase.
Asked about this I might say, ‘The goal of my actions was not to break the vase
but only to make her laugh, of course.’
In talking about goals in this way I am not talking about mental representations:
I am talking about actual and possible outcomes of an action, things which did or might 
have happened.
 
A goal is an outcome to which an outcome is directed.
 
\subsection{slide-3}
For a process to track a goal of an action is for how that process unfolds to nonaccidentally
depend in some way on whether that outcome is indeed a goal of the action.
 
\subsection{slide-4}
In this lecture I propose to introduce a developmental puzzle about goal tracking.
Along the way I will also introduce some theories of goal-tracking, none of which 
seems entirely adequate to the evidence.
 
\subsection{slide-5}
One last thing before I start: my focus is pure goal-tracking, that is, goal-tracking which
does not depend on any information about mental states.
 
I take pure goal-tracking to be a fundamental capacity, one which anchors abilities to 
understand others’ minds and underpins capacities to perform joint actions.
[Because it underpins our abilities to identify mental states and to generate predictions based on
ascriptions of mental states.]
 
\subsection{slide-2}
 
 
\section{When can infants first track goals?}
 
A variety of evidence indicates that 
infants may be able to track goals from three months of age (or earlier).
 
\subsection{slide-6}
When can infants first track goals to which actions are directed?
 
\subsection{slide-7}
Consider a much-replicated study by Woodward and colleagues.
 
\subsection{slide-8}
'Six-month-olds and 9-month-olds showed a stronger novelty response (i.e., looked longer) on new-goal trials than on new-path trials (Woodward 1998). That is, like toddlers, young infants selectively attended to and remembered the features of the event that were relevant to the actor’s goal.'
\citep[p.\ 153]{woodward:2001_making}
 
\subsection{slide-9}
Using a manipulation we’ll discuss later (‘sticky mittens’),
\citet{sommerville:2005_action} used this paradigm to show that even
three-month-olds can form expectations based on the goal of an action
(for another study with three-month-olds, see also \citealp{luo:2011_threemonthold}).
 
\subsection{slide-10}
The studies we have considered so far focus on infants’ abilities to identify the targets of actions.
They do not show that infants can identify the type of an action---for instance, whether it is a grasping or a pushing action.
 
\subsection{slide-11}
To say that infants can track  goals to which actions are directed implies that they can distinguish both the target and the type of an action.
So can infants also distinguish between two actions which are directed to the same target but differ in type?
 
\subsection{slide-12}
To answer this question, we would ideally have pairs of scenarios in which the target of an action
is kept constant while the type of action varies.
To the extent that subjects respond appropriately to the difference in type of action, we can be
confident that they can distinguish actions not just by their targets but also by their types.
 
This is illustrated here, where you are habituated to a grasping event and the test events are
(a) grasping but with novel kinematics (from a different angle), or (b) a novel goal event (pushing).
 
Unfortunately, as far as I know this has not yet been done.
But there are some studies which, although not intended to get at exactly the issue of whether
infants track goals and not just targets of actions, do indirectly shed light on this issue.
 
These studies demonstrate competence in goal-tracking from nine months of age, and give us no reason
to doubt that, in simple enough cases, infants might show competence in goal tracking much earlier.
 
\citet{Behne:2005dw} created just such pairs of contrasting scenarios.
In one of their contrasts, an experimenter holds a ball out for an infant to grasp and then
either ‘accidentally’ drops it or teasingly pulls it back.
So in each case there is a goal-directed action involving the ball, but in one case the goal
of the action is to pass the ball to the infant whereas in the other case the goal is to
tease the infant.
\citet[Study 2]{Behne:2005dw} found that nine-month-olds (but not six-month-olds)
consistently and appropriately discriminated between these scenarios by, for example,
banging more when the ball was ‘accidentally’ dropped than when it was teasingly retracted.
This and other research
(e.g. \citealp{ambrosini:2013_looking}) suggests that, at least from nine months of age,
infants can indeed distinguish both the type and target of a goal-directed action.
 
% \citet{ambrosini:2013_looking} is important: shows that type of grasp and not just target is tracked, implying that anticipatory looking is not merely perceptual animacy.
 
\subsection{slide-13}
Infants can track goals from nine months of age (or earlier).
 
 
 
\section{How are infants first able to track goals?}
 
The Teleological Stance provides a computational description of goal tracking.
It hinges on the principle that the goals of an action are those outcomes which the means are a
best available way of brining about.
Given this principle, we can infer goals from means by identifying which outcomes the means 
are a best available way of brining about.
But how is this computation performed?
 
\subsection{slide-14}
Infants can track goals from nine months of age (or earlier).
 
The question, of course, is how?
 
\subsection{slide-16}
So planning is the process of moving from goals to means,
whereas tracking goes in the reverse direction, from means to goals.
But what is common to the two is the relation between means and goals.
In both cases, planning and goal-tracking, the means that are adopted should be a best available
way of bringing the goal about.
 
\subsection{slide-18}
Note that this is not exactly an answer to our question,
How can infants track goals from nine months of age (or earlier)?
It provides what Marr would call a computational description.
 
That is, it provides a function
from 
facts about events and states of affairs that could be known without knowing which goals any
particular actions are directed to, nor any facts about particular mental states
to 
one or more outcomes which are the goals of an action.
 
Providing this function explains how pure goal-tracking is possible in principle.
 
But what we want to know, of course, is how infants (and adults) actually compute this function.
If this is (roughly) the function which computationally describes pure goal tracking,
what are the representations and processes involved in pure goal tracking?
 
\subsection{slide-19}
An we need to know how they compute to which outcome a means is the best available.
 
\subsection{slide-20}
The principles comprising the Teleological Stance are things we know or believe, and we are
able to track goals by making inferences from these principles and our beliefs about the
means someone is pursuing.
 
Infants and adults engaged in goal-tracking reason about 
to which outcome a means is the best available
in fundamentally the same way that you or I do when trying to work it out explicitly.
 
\subsection{slide-21}
[*TODO: illustrate with picture.]
 
\subsection{slide-22}
Although I don’t think they have written about it in quite the terms I use,
I take Gergely and Csibra to be endorsing the Simple View.
 
‘when taking the teleological stance one-year-olds apply the same
inferential principle of rational action that drives everyday mentalistic
reasoning about intentional actions in adults’
 
\subsection{slide-23}
The Simple View makes no use of the idea that there might be action perception.
 
Relation to Q3: no special system.
[Q3: Are these properties and happenings tracked by modality specific systems, or are they tracked by
special purpose perceptual mechanisms of their own (e.g. Liberman, 1967)/ cross-modal objects of
perception?]
 
Relation to Q6: teleological stance is continuous with mindreading (the latter merely involves
using an enriched model with includes mental states; the two can be parts of a single process).
[Q6: How do these processes relate to theory of mind abilities (e.g. Simmons et al., 2009)?]
 
\subsection{slide-24}
Infants can track goals from nine months of age (or earlier).
 
\subsection{goal\_tracking\_limit}
 
 
\section{A signature limit of infant goal tracking?}
 
A signature limit of a system is a pattern of behaviour the system exhibits which is both defective
given what the system is for and peculiar to that system.
There is some evidence that goal-tracking 
in infants (and adults) 
is limited by their abilities to act.
Could this be a signature limit of infant goal-tracking?
 
\subsection{slide-27}
To explain this I have to step back and show you something interesting about 
adults when they perform, and when they observe actions.
 
Performing actions (e.g. stacking blocks): you don't look at your hand but at
the block it will pick up, or, when holding a block, at the location where it will place a block.
That is, in acting the eyes move just ahead of the action.
 
\subsection{slide-28}
\citet{Flanagan:2003lm} showed that the same pattern occurs when adults observe another 
acting. 
In observing an action, the eyes move just ahead of the action.
 
This proactive gaze is important for our purposes because it can reveal goal-tracking ...
 
\citet{Flanagan:2003lm} showed that 
‘patterns of eye–hand coordination are similar when performing and observing a block stacking task’.
 
\subsection{slide-29}
Kanakogi \& Itakuar, 2011 show that abilities to grasp objects are correlated with
abilities to track the goal of a grasping action (as measured by proactive gaze).
 
x-axis is alpha, grasping angle. ‘An α angle value from 90 to 180° indicates that the infant
is engaged in a one-handed grasping action.’
 
‘The angle α is an index of the development of the one­handed grasping action and was
calculated by measuring the angle of a straight line de ned by the infant’s two hands (the
apex of the junction of the thumb and index nger) when crossed by an imaginary line
projecting frontally from the infant (Fig. 2b). If infants grasped for the objects with their
le hand, we reversed the red right­angled triangle from one side to the other side and
calculated the angle α in the same way. e angle α value of 90° corresponded to a perfect
alignment of the hands in a two­handed reach. Therefore, the angle α value deviates from 90°
towards 180°, and bigger angle α value indicates more mature one­handed grasping. If the
angle α was over 90°, the infant was considered to be engaged in a one­handed grasping
action.’
 
‘three action conditions: grasping hand (GH, n = 31), back of hand (BH, n = 31), and
mechanical claw (MC, n = 32). Pearson’s r reflects the partial correlation between timing of
gaze arrival at the goal and grasping ability in each condition after controlling for age.
Asterisk indicates statistical signi cance, P < 0.05; ns, not significant.’
 
\subsection{slide-30}
Further, changing from a bodily action to the operation of a mechanical claw
(say) undermines the goal tracking effect.
 
So  Kanakogi \& Itakuar, 2011 make two points:
(i) goal-tracking depends on action capabilities; and 
(ii) only works for events involving biomechanically similar affectors
 
\subsection{slide-32}
Needham et al, 2002 showed that putting ‘sticky mittens’ on 3-month-old infants 
(for 10-14 play sessions of 10 minutes each) resulted in their spending 
more time visually and manually exporing novel objects.
 
\subsection{slide-33}
In this study, I think infants wore the mittens for just 200 seconds
(so the play sessions were much shorter than in Needhman et al, 2002).
 
\subsection{slide-34}
The observation was based on this study, which we saw earlier
 
\subsection{slide-37}
The results show that infants who played wearing the mittens first
were more attentive to the goal.
 
From at least three months of age, some of infants’ abilities to identify
the goals of actions they observe are linked to their abilities to perform
actions \citep{woodward:2009_infants}.
 
But one potential objection to this study concerns observation vs performance.
The infants who played wearing sticky mittens first had spent longer observing
actions by the time it came to the violation of expectations trial.
Could it be observation of action (including one’s own) rather than performance
that matters?
 
\subsection{slide-38}
nb something gets stuck to the mittens; it's not really grasping!
 
\subsection{slide-39}
To address this issue, \citet{sommerville:2008_experience} did a study in 
which one group had observation while the other group had performance.
The participants were 10-month-old infants this time.
 
The materials were a bit different: so that training vs observation could 
be as similar as possible with respect to the causal structure exposed,
there was a hook to get an object.
 
\subsection{slide-40}
The results show that infants with the training paid attention to the
distal goal (choice of toy) whereas those without paid attention to the
choice of cane.
 
\subsection{slide-41}
Further support for a link between action performance and goal tracking
comes from a developmental study by Ambrosini et al which studied whether proactive
gaze in infants is influenced by pre-shaping of the hand, and, in particular, 
whether it is influenced by precision grips.
 
\subsection{slide-43}
By using no shaping (a fist), Ambrosini et al could treat sensitivity
to whole-hand grasp and precision grip separately.
 
\subsection{slide-44}
‘infants’ ability to perform specific grasping actions with fewer fingers directly predicted the degree with which they took advantage of the availability of corresponding pre-shape motor information in shifting their gaze towards the goal of others’ actions’ \citep[p.~6]{ambrosini:2013_looking}.
 
\subsection{slide-45}
In infants (and adults), 
goal-tracking is limited by their abilities to act.
 
Why is this true?
Why is goal-tracking in infants (and adult) limited by their abilities to act?
On the Simple View, goal tracking is a matter of thinking and reasoning about the 
best means to perform an action. 
On this View, there’s no obvious reason why 
your goal-tracking should be limited by your abilities to act in this way.
 
Although I can’t jump over a house, I can perfectly well think about different ways
to do so and distinguish better and worse approaches, at least to some extent.
 
\subsection{motor\_theory\_goal\_tracking}
 
 
\section{The Motor Theory of Goal Tracking}
 
How do infants (and adults) track the goals of others’ actions?
According to the Motor Theory of Goal Tracking, it is sometimes* by 
means of motor processes.
More carefully, the Motor Theory of Goal Tracking consists of these claims: 
(1) in action observation, possible outcomes of observed actions are represented motorically;
(2) these representations trigger motor processes much as if the observer were performing actions
directed to the outcomes;
(3) such processes generates predictions;
(4) a triggering representation is weakened if the predictions it generates fail.
The result is that, often enough, the only only outcomes to which the observed action is a means
are represented strongly.

(*‘sometimes’ because the Motor Theory is part of a dual-process account of goal-tracking.)
 
\subsection{slide-47}
Suppose you are reaching for, grasping, transporting and then placing a pen. Performing even
relatively simple action sequences like this involves satisfying many constraints that cannot
normally be satisfied by explicit practical reasoning, especially if performance is to be rapid and
fluent. Rather, such performances require motor representations.
These representations are paradigmatically involved in preparing, executing and monitoring actions.%
\footnote{%
See \citet{wolpert:1995internal, miall:1996_forward, jeannerod:1998nbo, zhang:2007_planning}.
Note that motor representations sometimes occur in an agent who has prepared an action and is required (as it turns out) not to perform it: although she has prevented herself from acting, motor representations specifying the action persist, perhaps because they are necessary for monitoring whether prevention has succeeded \citep{bonini:2014_ventral}.
}
But they also live a double life. Motor representations concerning a particular type of action are
involved not only in performing an action of that type but also sometimes in observing one. That is,
if you were to observe Ayesha reach for, grasp, transport and then place a pen, motor representations
would occur in you much like those that would also occur in you if it were you---not Ayesha---who was
doing this.
 
Converging evidence for this assertion comes from a variety of methods and measures;
but I won’t mention any of that here.
 
\subsection{slide-49}
mTgt is an alternative to the Simple View.
The idea is that pure goal-tracking involves motor processes rather than thinking
and reasoning about goals.
 
But how could motor processes enable goal tracking?
 
\subsection{slide-50}
At this point it is helpful to think about speech.
To utter a phoneme is to produce complex coordinated movements of the 
lips, larynx, tongue and velum.
Further, how these should be moved and what sound should be produced depends on
many factors including phonemic context, rate of speech and prosody.
 
So uttering a phoneme is unlike pressing a key on a piano keyboard:
it not a matter of producing a particular sound
but of performing a complex, goal-directed action.
 
We know from Riikka Möttönen’s research that there is striking new
evidence for what is sometimes called the Motor Theory of Speech Perception.
 
So speech perception is special case of goal-tracking.
The perceiver’s task is to recover the goal to which the utterance of a phoneme is directed.
 
mTgt is simply a generalisation of the Motor Theory of Speech Perception
 
In performing an action, there are motor processes which compute means given goals.
The same processes can be used when observing an action for tracking goals.
 
\subsection{slide-51}
Goal-tracking is acting in reverse.
-- in action observation, possible outcomes of observed actions are represented
-- these representations trigger planning as if performing actions directed to the outcomes
-- such planning generates predictions
-- a triggering representation is weakened if its predictions fail
The result is that the only only outcomes to which the observed action is a means
are represented strongly.
 
There is evidence that a motor representation of an outcome can cause a determination of which
movements are likely to be performed to achieve that outcome \citep[see, for
instance,][]{kilner:2004_motor, urgesi:2010_simulating}. Further, the processes involved in
determining how observed actions are likely to unfold given their outcomes are closely related,
or identical, to processes involved in performing actions.
This is known in part thanks to studies of how observing actions can facilitate performing
actions congruent with those observed, and can interfere with performing incongruent actions
\citep{
	brass:2000_compatibility, 
	craighero:2002_hand, 
	kilner:2003_interference, 
	costantini:2012_does}. 
Planning-like processes in action observation have also been demonstrated by measuring observers' predictive gaze.  If you were to observe just the early phases of a grasping movement, your eyes might jump to its likely target, ignoring nearby objects \citep{ambrosini:2011_grasping}. These proactive eye movements resemble those you would typically make if you were acting yourself \citep{Flanagan:2003lm}. 
Importantly, the occurrence of such proactive eye movements in action observation depends on your
representing the outcome of an action motorically; even temporary interference in the observer's
motor abilities will interfere with the eye movements \citep{Costantini:2012fk}.
These proactive eye movements also depend on planning-like processes; requiring the observer to
perform actions incongruent with those she is observing can eliminate proactive eye movements
\citep{Costantini:2012uq}. This, then, is further evidence for planning-like motor processes in
action observation.
 
So observers represent outcomes motorically and these representations trigger planning-like processes
which generate expectations about how the observed actions will unfold and their sensory consequences.
Now the mere occurrence of these processes is not sufficient to explain why, in action observation,
an outcome represented motorically is likely to be an outcome to which the observed action is
directed.
 
To take a tiny step further, we conjecture that, in action observation, \textbf{motor representations of
outcomes are weakened to the extent that the expectations they generate are unmet}
\citep[compare][]{Fogassi:2005nf}.
A motor representation of an outcome to which an observed action is not directed is likely to
generate incorrect expectations about how this action will unfold, and failures of these
expectations to be met will weaken the representation.
This is what ensures that there is a correspondence between outcomes represented motorically in
observing actions and the goals of those actions.
 
\subsection{slide-52}
The mTgt cannot be a full theory of goal-tracking in adults, of course.
Instead we need a dual process theory of goal-tracking.
 
\subsection{slide-53}
But what is a dual process theory of goal-tracking?
I’m so glad you asked.
It’s very simple ...
 
\subsection{slide-54}
‘proactive gaze’ and ‘explicit judgement’ are variables whose values represent
whether there is a proactive gaze or explicit judgement, and what is it to or about.
Likewise, ‘motor process’ is a variable whose values represent ...
 
The lines depict how the variables are causally related. I’ve used thick and thin
lines informally, to indicate weaker and stronger influences.
 
\subsection{slide-57}
The dual-process theory of goal-tracking makes perfect sense of development.
It says that what we observe in six- and nine-month-olds is motor-based goal-tracking.
Presumably the more flexible, reasoning-based goal-tracking processes emerge some time
later in development.
 
\subsection{slide-58}
Infants can track goals from nine months of age (or earlier).
 
The Motor Theory of Goal Tracking
 
Relation to Q3: special system, namely the motor system.
 
Relation to Q6 [How do these processes relate to theory of mind abilities (e.g. Simmons et al.,
2009)?]:
It looks like, in endorsing mTgt, we are postulating a discrepancy between goal-tracking and
theory of mind abilities, since theory of mind abilities are perhaps unlikely to depend on motor
processes in the way in which goal-tracking does. 
However, we should probably be cautious here; in pilot research with Jason Low, we have some
preliminary evidence that some theory of mind abilities may be more closely tied to motor processes
than anyone has yet imagined.
 
\subsection{slide-60}
Important that 
mTgt is not an alternative to the Teleological Stance but to it plus the claim about reasoning.
 
Simple View and mTgt do not differ on the relation between means and goals that is to be 
computed in pure goal-tracking.
The Simple View and mTgt differ only on what processes is responsible for identifying which outcome or
outcomes the observed means is a best available way of achieving.
 
\subsection{slide-66}
There’s just one small problem. It’s not quite true to say that 
infants goal-tracking is limited by action ability.
Some of it is, but some of it isn’t ...
 
\subsection{slide-61}
 
 
\section{A Developmental Puzzle about Goal-Tracking}
 
In infants,
it appears that
some,
but not all,
goal-tracking is limited by their abilities to act.
This contradicts both the Simple View and the Motor Theory of Goal Tracking.

A further puzzle arises from the fact that goal-tracking sometimes manifests in dishabitution or
pupil dilation but not proactive gaze.
How can we explain, in a principled way, why there should be discrepancies between these measures?
 
\subsection{slide-67}
Earlier I mentioned this experiment, which shows that infants 
fail to track goals involving things which would appear to them to be 
biomechanically non-agent-like events.

But ...
 
\subsection{slide-68}
\#source 'research/teleological stance -- csibra and gergely.doc'
 
\#source 'lectures/mindreading and joint action - philosophical tools (ceu budapest 2012-autumn fall)/lecture05 actions intentions goals'
 
\#source 'lectures/mindreading and joint action - philosophical tools (ceu budapest 2012-autumn fall)/lecture06 goal ascription teleological motor'
 
When do human infants first track goal-directed actions and not just movements?
 
Here's a classic experiment from way back in 1995.
 
The subjects were 12 month old infants.
 
They were habituated to this sequence of events.
 
\subsection{slide-69}
There was also a control group who were habituated to a display like this
one but with the central barrier moved to the right, so that the action
of the ball is 'non-rational'.
 
\subsection{slide-70}
For the test condition, infants were divided into two groups.  One saw a new action, ...
 
... the other saw an old action.
 
Now if infants were considering the movements only and ignoring information about the goal, the 'new action' (movement in a straight line) should be more interesting because it is most different.
 
But if infants are taking goal-related information into action, the 'old action' might be unexpected and so might generate greater dishabituation.
 
\subsection{slide-72}
You might say, it's bizarre to have used balls in this study, that can't show us anything about infants' understanding of action.
 
But adult humans naturally interpret the movements of even very simple shapes in terms of goals.
 
So using even very simple stimuli doesn't undermine the interpretation of these results.
 
\subsection{slide-73}
Consider a further experiment by \citet{Csibra:2003jv}, also with
12-month-olds. This is just like the first ball-jumping experiment
except that here infants see the action but not the circumstances in
which it occurs. Do they expect there to be an object in the way behind
that barrier?
 
\subsection{slide-75}
Here then is the puzzle about development that I mentioned this talk was about:
 
In infants, 
it appears that 
some, 
but not all, 
goal-tracking is limited by their abilities to act.
 
Let me recap how we got here and why this is puzzling.
 
I started by asking, How 9-month-olds track can goals?
The Simple View offers one answer: 
the principles comprising the Teleological Stance are things they know or believe, and they are able
to track goals by making inferences from these principles.
 
I suggested that the Simple View should be rejected because it cannot explain why infants’
abilities to track goals are limited by their abilities to perform actions.
 
At least, mTga provides a better alternative to the Simple View.
According to mTga,
Infants’ pure goal-tracking depends on the double life of motor processes.
 
mTga correctly predicts that infants’ pure goal-tracking should be limited by infants’
abilities to act ...
 
\subsection{slide-76}
But, unfortunately, there appear to be cases in which infants’ pure goal-tracking
is not limited by their abilities to act, and this is contrary to mTga.
 
The puzzle, then, to explain how infants can track goals if neither the Simple View
nor mTga is correct.
 
Even worse, 
there’s another, apparently unrelated puzzle to explain too ...
 
\subsection{slide-77}
Daum et al created a modified version of Woodward’s paradigm which allowed them to measure
both anticipatory looking and dishabituation.
 
where researchers have measured two different responses to a single
scenario, anticipatory looking and either dishabituation or pupil dilation.
Generally, they appear to find evidence for goal tracking in dishabituation
or pupil dilation but not in anticipatory looking. 
(This is true of Daum et al, 2012; and Gredeback and Melinder, 2010.)
 
Why the discrepancy?
This is another question we can’t answer with the Motor Theory of Goal
Tracking.
 
\subsection{slide-78}
[skip -- just here in case need for discussion; shows that anticipatory looking
to cartoon fish takes years to develop]
 
\subsection{slide-79}
Here then is another puzzle about development:
 
In infants, 
it appears that 
some, 
but not all, 
goal-tracking is limited by their abilities to act ...
 
... and that goal-tracking sometimes manifests in dishabitution or pupil dilation but not proactive gaze.
 
The \textbf{second puzzle} is how to explain, in a principled way, why there should be discrepancies
between these measures. We cannot do this by invoking mTgt because on mTgt, proactive gaze is a
case in which goal-tracking is paradigmatically manifest.
 
\subsection{slide-80}
I want to respond by arguing that not everything which appears to be goal-tracking in infants
actually is goal-tracking.
 
\subsection{slide-81}
So I will be arguing that the puzzle is merely apparent.
The appearance is due to the fact that we do not carefully enough distinguish tracking
the target of an action from tracking the goal of an action.
 
\subsection{perceptual\_animacy}
 
 
\section{Perceptual Animacy}
 
Perceptual animacy is
the detection by broadly perceptual processes of animate objects and their targets
\citet[e.g.][]{gao:2009_psychophysics}.
Could its existence provide us with a solution to the developmental puzzle about 
goal tracking?
 
\subsection{slide-83}
\emph{perceptual animacy},
the detection by broadly perceptual processes of animate objects and their targets.
 
Evidence for the existence of perceptual animacy comes from a variety of studies 
where adults are given a simple visual task such as identifying which circle is the ‘wolf’
and which the ‘sheep’, or, in another experiment, moving the sheep around in order to 
avoid being caught by the wolf.
 
It is also possible to examine how having a sheep and a wolf affects how attention is allocated by
measuring how well participants can detect probes placed on different shapes.
 
 
The literature on perceptual animacy mostly confounds it with goal tracking.
But there are two key differences.
 
First, perceptual animacy is a broadly
perceptual phenomenon.
 
Second,
the perceptual detection of animacy is clearly distinct from goal tracking
for it involves detecting relations between objects in motion rather than
outcomes to which actions are directed.
Relatedly, it does not involve sensitivity to the type of action.
And the perceptual detection of animacy appears to depend on simple cues
and heuristics and is unlikely to be correctly described by the
Teleological Stance.
So perceptual animacy does not involve tracking goals to which an action is
directed: it is a matter of tracking objects to which animate movements are
directed.
 
\subsection{slide-84}
\emph{{perceptual animacy}},
the detection by broadly perceptual processes of animate objects and their targets.
To illustrate, consider a groundbreaking experiment by \citet[experiment 1]{gao:2009_psychophysics}.
Adults were shown a display which contained some moving circles.
In some cases the circles moved independently of each other, but in other cases there was a ‘wolf’
which chased a ‘sheep’ with varying degrees of subtlety.
The adults’ task was simply to detect the presence of a wolf.
\citeauthor{gao:2009_psychophysics} established that adults can do this providing the chasing is not too subtle.
In further experiments, they also showed that adults’ abilities to perceptually detect chasing
depend on several cues including whether the chaser ‘faces’ its target (‘directionality’) and how
directly the chaser approaches its target (‘subtlety’).
The detection of animacy appears to be a broadly perceptual phenomena since it depends on areas of
the brain associated with vision and influences how perceptual attention is allocated
\citep{scholl:2013_perceiving} irrespective of your beliefs and intentions
\citep{buren:2016_automaticity}.
 
\subsection{slide-86}
[This comparison is a bit confused because pure-goal tracking is the kind of thing 
that can be (and is) achieved by way of different mechanisms (e.g. reasoning; motor processes),
whereas perceptual animacy is a mechanism.]
 
The perceptual detection of animacy resembles goal tracking, for it involves detecting a relation
between a chaser and its target.
However the perceptual detection of animacy is clearly distinct from goal tracking.%
\footnote{%
Contrast \citet{schlottmann:2010_goal,Scholl:2000eq} who all claim that perceptual animacy is a matter of, or involves, tracking goals.
}
For one thing, it involves detecting relations between objects in motion rather than outcomes to
which actions are directed.
Relatedly, it does not involve sensitivity to the type of action.
Finally, the perceptual detection of animacy appears to depend on simple cues and heuristics and is
unlikely to be correctly described by the Teleological Stance.
For these reasons, we should distinguish the perceptual detection of animacy from tracking goals to which actions are directed.
 
\subsection{slide-89}
Conjecture


          
In 9-month-olds,
          


          
all pure goal-tracking is explained by the Motor Theory;
          


          
appearances that goal-tracking is not limited by their abilities to act
are due to perceptual animacy.
          

 
\subsection{slide-90}
Predictions
            


            
Where 9-month-olds appear to be tracking goals 
in ways not limited by their abilities to act,
they will be subject to signature limits of perceptual animacy
(e.g. subtlety, directionality);


            
and the processes underlying their abilities will be broadly perceptual.
            

 
\subsection{slide-91}
[Haven’t slides; will use printout from their paper]
 
\subsection{slide-92}
How does the conjecture help with the puzzles about development?
 
In infants, 
it appears that 
some, 
but not all, 
goal-tracking is limited by their abilities to act ...
 
... and that goal-tracking sometimes manifests in dishabitution or pupil dilation but not proactive gaze.
 
The second puzzle was a dissociation between dishabituation and pupil dilation on the one
hand and proactive gaze on the other.
The conjecture I have formulated suggests a solution: perhaps perceptual animacy is sometimes
responsible for pupil dilation and dishabituation; but never responsible for proactive gaze.
 
\subsection{slide-93}
The first puzzle was the appearance of cases of goal-tracking in infancy which is not
limited by infants’ abilities to act.
The conjecture I have proposed suggests a solution to this too: those cases 
are underpinned by perceptual animacy and so not genuine cases of goal-tracking at all.
 
\subsection{slide-95}
Pupil dilation at 6 and 12-months of age; unlike proactive gaze, it  does not 
correlate with experience of being fed; but it isn’t plausibly perceptual animacy either.
(Why would perceptual animacy predict a difference between spoon going to mouth or hand?)
Instead it may be a consequence of associative learning about the typical sequence
of chunks of actions.
 
\subsection{slide-96}
My question was, How can infants in the first 9 months of life track goals?
Given that infants’ goal-tracking is limited by their abilities to perform actions,
I proposed to answer this question by appeal to the mTgt:
Infants’ pure goal-tracking, like much of adults’, depends on the double life of motor processes.
 
This yielded a simple dual-process theory of goal-tracking.
 
But this is much too simple, as
findings on infants’ goal-tracking in the first 9 months of life confronts us with a puzzle.
For they appear to show that, in many cases, infants can track the goals of actions 
performed by cartoon fish and spheres; and these are clearly actions that no infant (or adult)
could perform.
 
Apparantly, then, we must reject the mTgt. But if we do that, it seems we lack any way of 
explaining why infants’ goal-tracking is so often limited by their abilities to act.
 
The solution, I suggested, is to invoke perceptual animacy.
That is what led me to this conjecture ...
 
\subsection{slide-97}
Conjecture


          
In 9-month-olds,
          


          
all pure goal-tracking is explained by the Motor Theory;
          


          
appearances that goal-tracking is not limited by their abilities to act
are due to perceptual animacy.
          

 
This conjecture is quite bold, but I think it’s one worth betting on partly because,
as I suggested, it generates readily testable predictions.
 
This conjecture also has another virtue.
Goal-tracking is so fundamental to social cognition and joint action that it is 
likely to depend on a rich mix of many kinds of processes, including associative learning.
We need ways to identify these processes, to distinguish their limits and to understand their 
synergies.
Distinguishing motor processes which support goal-tracking by realising the Teleological Stance 
from perceptual animacy is a step towards meeting this need.
 
So how does this bear on the conference topic,
Action Perception?
In short, there isn’t any.
At least not if actions are goal-directed actions.
 
Why do I say this?  Not because I can rule out the possibility of action perception, of course.
But goal-tracking is like speech perception in this sense: in both cases, the best supported,
most carefully developed theories are not about perceptual processes but about motor processes.
 
\subsection{slide-98}
That, anyway, is my conclusion.
The double life of motor processes underpins humans’ earliest 
pure goal-tracking, and so provides a basis for the many and varied forms of social cognition,
including mindreading, which depend on pure goal-tracking.
 
Many will disagree with this.
What I hope we will agree about, though, is that solving the developmental puzzle
about pure-goal tracking is essential if we care at all about the abilities
which underpin humans’ capacities for social cognition and joint action.
 

    

%--- end paste
%---------------





\bibliography{$HOME/endnote/phd_biblio}



\end{document}
