 %!TEX TS-program = xelatex
%!TEX encoding = UTF-8 Unicode

%\def \papersize {a5paper}
\def \papersize {a4paper}
%\def \papersize {letterpaper}

%\documentclass[14pt,\papersize]{extarticle}
\documentclass[12pt,\papersize]{extarticle}
% extarticle is like article but can handle 8pt, 9pt, 10pt, 11pt, 12pt, 14pt, 17pt, and 20pt text

\def \ititle {Origins of Mind: Lecture Notes}
\def \isubtitle {Lecture 01}
%comment some of the following out depending on whether anonymous
\def \iauthor {Stephen A.\ Butterfill}
\def \iemail{s.butterfill@warwick.ac.uk% \& corrado.sinigaglia@unimi.it
}
%\def \iauthor {}
%\def \iemail{}
%\date{}

%\input{$HOME/Documents/submissions/preamble_steve_paper4}
\input{$HOME/Documents/submissions/preamble_steve_lecture_notes}

%no indent, space between paragraphs
\usepackage{parskip}

%comment these out if not anonymous:
%\author{}
%\date{}

%for e reader version: small margins
% (remove all for paper!)
%\geometry{headsep=2em} %keep running header away from text
%\geometry{footskip=1.5cm} %keep page numbers away from text
%\geometry{top=1cm} %increase to 3.5 if use header
%\geometry{bottom=2cm} %increase to 3.5 if use header
%\geometry{left=1cm} %increase to 3.5 if use header
%\geometry{right=1cm} %increase to 3.5 if use header

% disables chapter, section and subsection numbering
\setcounter{secnumdepth}{-1}

%avoid overhang
\tolerance=5000

%\setromanfont[Mapping=tex-text]{Sabon LT Std}


%for putting citations into main text (for reading):
% use bibentry command
% nb this doesn’t work with mynewapa style; use apalike for \bibliographystyle
% nb2: use \nobibliography to introduce the readings
\usepackage{bibentry}

%screws up word count for some reason:
%\bibliographystyle{$HOME/Documents/submissions/mynewapa}
\bibliographystyle{apalike}


\begin{document}



\setlength\footnotesep{1em}






%---------------
%--- start paste

\title {Philosophical Psychology \\ Lecture 05: Do Humans Perceive Others’ Feelings}



\maketitle

\subsection{title-slide}
subtitle is:
Prospects for the Hypothesis That We Perceive Mental States,
and How Some Limits on What Can Be Perceived Are Overcome through Simple Forms of Social Interaction.

This talk is about a simple question.
Do humans every perceptually expeirence any of another’s mental states?
What evidence would help us to answer this question?
At the very end, I will switch from perception to interaction,
and consider what we might gain by
shifting from merely observing expressions of emotion to
jointly expressing an emotion as in sharing a smile.

\subsection{slide-3}
Here are three ways of knowing: inference, testimony and perception.

\subsection{slide-4}
Where ‘it’ is an ordinary physical object like my favourite pen,
each of these three identifies a process by which knowledge could be acquired.

(This is not to say that talk about reasoning invariably picks out a process (cf Alvarez),
just that there is a process.)

There seems to be a clear contrast the first and last ways of coming to know about it.
(Philosophers debate about whether the first and second are genuinely different routes
to knowledge.)

\subsection{slide-5}
Even if, as do I, you think perception involves inference-like processes, there is still
a contrast between inferring and perceiving.

For instance, suppose you know that I am all but inseparable from my favourite pen.
Then when you see me arrive you can infer that my pen is here too.
Contrast this with simply seeing my pen on the table.
In one case we see something other than the pen, namely me, and inferring that the
pen is here; in the other case you are seeing the pen itself.

This is the contrast I need in what follows.

\subsection{slide-8}
Are mental states the kinds of thing we can come to know about through
perceiving them?
I will focus on emotions like joy.

\subsection{slide-9}
[not relevant yet:] ‘(3) a losing face on a winning body’

Emotions are sometimes expressed bodily and vocally.
Here is someone who has just won or lost a tennis match, so is feeling
joy or anger.  Maybe you can tell which?

Here is a natural thought:
Bodily and vocal expressions of emotion
enable us to perceptually experience the expressed emotions.

This natural thought fits with ordinary talk about mental states.
Imagine yourself at a tennis match.
Reporting the event afterwards, you might say that you saw his ecstacy
at winning.
It might seem that his ecstacy is as plainly visible as the hair on his head.

\subsection{slide-10}
There are scientists and philosophers who have placed a lot of weight
on verbal reports.
They are satisfied that if people use mental state terms to describe what they
perceive, then they perceive those mental states.
But I think this can’t be right.

Go back to the tennis match.
Reporting the event afterwards you might say not only that you saw his ecstacy
but also that you saw him win.
I take it you can't literally perceptually experience winning or losing
in the sense that, arguably, you can perceptually experience movement or changes in colour.
After all, winning or loosing is a matter of rules and conventions.

\subsection{slide-11}
So we can’t straightforwardly take what people say as a guide to what they perceive.

Here you might object that we can perceptually experience things like winning or losing,
so there is no reason to question verbal reports.

But note that taking this line amounts to rejecting the claim I started with.

\subsection{slide-12}
I claimed that there is a contrast between seeing an indicator and inferrring the
presence of an object

If there is a contrast here, what we colloquially call seeing someone win must be
a case of perceiving indicators and inferring winning.
So to say that we can straightforwardly infer that people perceive emotions from
their verbal reports is to reject the existence of the very contrast that allows
us to make sense of the question.

\subsection{slide-13}
‘We sometimes see aspects of each others’ mental lives, and thereby come to have non-inferential knowledge of them.’

\subsection{categorical\_perception\_emotion}


\section{Categorical Perception \& Emotion}

\subsection{slide-16}
Categorical perception is perhaps most easily understood from the case of colour.

[For later: colour is doubly relevant because there's been serious debate about
whether it's possible to perceive categorical
colour properties despite copious verbal reports.
This case shows how evidence can bear on questions about phenomenology,
although I won’t be talking about that here (probably).]

\subsection{slide-17}
fix initial system of categories

measure disciminatory responses

observe between- vs within-category differences

exclude non-cognitive explanations for the differences


greater discrimination between than within catgeories indicated that the inital
system of categories may be having some influence on whatever underpins
the responses.

Do any perceptual processes in humans discriminate stimuli according to the expressions
of emotion they involve?
That is, do humans have \emph{categorical perception} of expressions of emotion?

\subsection{slide-18}
Assume that we as theorists have a system which allows us to categorise
static pictures of faces and other stimuli according to which emotion we
think they are expressing: some faces are happy, others fearful, and so on.

There is a way of categorising static pictures of faces and other stimuli according to which emotion
someone might think they are expressing: some faces are happy, others fearful, and so on
From five months of age,
or possibly much earlier \citep{field:1982_discrimination},
through to adulthood, humans are better at distinguishing faces when they
differ with respect to these categories than when they do not
\citep{Etcoff:1992zd,Gelder:1997bf,Bornstein:2003vq,Kotsoni:2001ph,cheal:2011_categorical,hoonhorst:2011_categoricala}.

\subsection{slide-19}
To illustrate, consider these pictures of faces.
The idea is this.

With respect to all features apart from the expression of emotion, each face picture differs from its neighbours no more than any other picture differs from its neighbours.

\subsection{slide-20}
Most neighbouring pairs of face pictures would be relatively hard to distinguish,

\subsection{slide-21}
especially if they were not presented side-by-side.

\subsection{slide-22}
But most people find one pair of neighbouring face pictures  relatively easy to distinguish---you may notice this yourself.

\subsection{slide-27}
What underlies these patterns of discrimination?
Several possibilities that would render them uninteresting for our purposes can be ruled out.

The patterns of discrimination do not appear to be an artefact of linguistic labels
(\citealp{sauter:2011_categorical}; see also \citealp{laukka:2005_categorical}, p.\ 291),%

%
\footnote{
Puzzlingly, experiments by \citet{fugate:2010_reading} using photos of chimpanzee faces with human subjects are sometimes cited as evidence that categorical perception of expressions of emotion depends on, or can be modulated by, the use of verbal labels for stimuli (e.g.\ \citealp[p.\ 288]{barrett:2011_context}; \citealp[p.\ 315]{gendron:2012_emotion}).
Caution is needed in interpreting these findings
given that there may be differences in the ways  humans process human and chimpanzee faces.
In fact, what \citeauthor{fugate:2010_reading}'s findings show may be simply that `human viewers do not show [categorical perception] for the chimpanzee facial configurations used in their study' \citep[p.\ 1482]{sauter:2011_categorical}.
}
%
nor of the particular choices subjects in these experiments are presented with \citep{bimler:2001_categorical,fujimura:2011_categorical}.
Nor are the patterns of discrimination due to narrowly visual features of the stimuli used \citep{sato:2009_detection}.

We can be confident, then, that the patterns of discrimination probably reflect one or more processes which categorises stimuli by expression of emotion.

\subsection{slide-28}
Examples of stimuli used (they had 200 ish faces) by Batty and Taylor in their ERP study.

\subsection{slide-29}
Don't have details, this is just to stress it's early (around 200ms) and plausibly automatic.

‘at a mean latency of 140 ms) the N170 showed both amplitude and latency
modulation differentially with emotional expressions. ...
[BUT] Whether this is due presently to low-level stimulus factors or to the use of emotional faces
is still to be determined’ \citep[p.~616]{batty:2003_early}.

‘As the task did not require the subjects to focus on particular emotional expressions ...
these data suggest an early automatic encoding of emotional facial expression’ \citep[p.~616]{batty:2003_early}.

At least some of the processes underpinning categorical perception of facial expressions
of emotion are rapid (occurring within roughly 200 milliseconds of a stimulus' appearance),
pre-attentive \citep{vuilleumier:2001_emotional} and automatic in the sense that whether they occur
is to a significant degree independent of subjects' tasks and motivations \citep{batty:2003_early}.%

\subsection{slide-30}
But are any of the processes that categorise stimuli by expression of emotion perceptual?
%That is, are the observed abilities to discriminate  expressions of emotion ever based on perceptual processes?
Answering this question is complicated by the fact that many parts of the brain are involved \citep{adolphs:2002_recognizing,vuilleumier:2007_distributed}.
There is evidence that both the amygdala \citep{harris:2012_morphing,harris:2014_dynamic} and also some cortical structures \citep{batty:2003_early}
respond categorically to  expressions of emotion;
and that intervening in the operations of the somatosensory cortex can impair categorisation (\citealp{pitcher:2008_transcranial}; see also \citealp{banissy:2011_superior}).
To my knowledge, so far it is only for happy and fearful stimuli that we have direct evidence
from both neurophysiological \citep{Campanella:2002aa} and behavioural measures \citep{williams:2005_looka}
of categorisation occurring in perceptual processing.
So while the evidence is not conclusive,
there is converging evidence that some perceptual processes categorise stimuli including faces by  expression of emotion.
Humans may have categorical perception not only for speech, colour, orientation and other properties but also for expressions of emotion.

\subsection{slide-31}
Earlier I asked, what evidence could bear on the perceptual hypothesis?
Now we have a partial answer.

\subsection{slide-32}
Part of the evidence relevant to the perceptual hypothesis is evidence that
humans can categorically perceive expressions of emotion.

But what does the evidence from studies of categorical perception
tell us about the truth or falsity of the perceptual hypothesis ?

\subsection{slide-33}
Just here it is natural to take a sceptical line and claim ...

\subsection{slide-34}
Just here we have to be extremely careful.
We have to ask, What are the objects of categorical perception?
That is, What Are the Perceptual Processes Supposed to Categorise?

\subsection{objects\_of\_categorical\_perception}


\section{The Objects of Categorical Perception}

\subsection{slide-36}
standard view: fixed expressions linked to emotional categories

\subsection{slide-37}
But are the things categorised by perceptual processes facial configurations?
This view faces a problem.
There is evidence that
the same facial configuration can express intense joy or intense  anguish depending on the posture of the body it is attached to,
and, relatedly, that humans cannot accurately determine emotions from spontaneously occurring
(spontaneously occurring---i.e.\ as opposed to acted out)
facial configurations \citep{motley:1988_facial,aviezer:2008_angry,aviezer:2012_body}.
These and other findings, while not decisive, cast doubt on the view that categories of emotion are associated with categories of facial configurations \citep{hassin:2013_inherently}.

\subsection{slide-38}
Aviezer et al's puzzle:
Given that  facial configurations are not diagnostic of emotion, why  are they categorised by perceptual processes?

This evidence makes the findings we have reviewed on categorical perception puzzling.
Given that the facial configurations are not diagnostic of emotion,
why are they categorised by perceptual processes?%
\footnote{
Compare \citet[p.\ 1228]{aviezer:2012_body}:
`although the faces are inherently ambiguous, viewers experience illusory affect and erroneously
report perceiving diagnostic affective valence in the face.'
}
This question appears unanswerable as long as we retain the assumption---for which, after all, no
argument was given---that the things categorical perception is supposed to categorise are facial
configurations.

\subsection{slide-39}
But if we reject this assumption, what is the alternative?

\subsection{slide-42}
Compare expressing an emotion by, say, smiling or frowning,  with articulating a phoneme.

\subsection{slide-43}
Both have a communicative function (on expressions of emotion, see for example \citealp{blair:2003_facial,sato:2007_spontaneous})
and both are categorically perceived,
but the phonetic case has been more extensively investigated.

\subsection{slide-44}
Variations due to coarticulation, rate of speech, dialect and many other factors mean that isolated acoustic signals are not generally diagnostic of phonemes:
in different contexts, the same acoustic signal might be a consequence of the articulation of any of several phonemes.

\subsection{slide-45}
So here there is a parallel between speech and emotion.
Much as isolated facial expressions are not diagnostic of emotions (as we saw a moment ago), isolated acoustic signals are plausibly not diagnostic of phonetic articulations.

\subsection{slide-46}
Why then are isolated acoustic signals---which rarely even occur outside the lab---categorised by perceptual or motor processes at all?
To answer this question we first need a rough idea of what it is to articulate a phoneme.
Articulating a phoneme involves making coordinated movements of the lips, tongue, velum and larynx.
How these should move depends in complex ways on numerous factors including phonetic context \citep{Browman:1992da,Goldstein:2003bn}.
In preparing for such movements, it is plausible that the articulation of a particular phoneme is an outcome represented motorically,
where this motor representation coordinates the movements and normally does so in such a way as to increase the probability that the outcome represented will occur.

\subsection{slide-47}
This implies that the articulation of a particular phoneme,
although probably not an intentional action,
is a goal-directed action whose goal is the articulation of that phoneme.

(On the link between  motor representation and goal-directed action, see \citealp{butterfill:2012_intention}.)

Now some hold that the things categorised in categorical perception of speech are not sounds or movements (say) but rather these outcomes---the very outcomes in terms of which speech actions are represented motorically (\citealp{Liberman:2000gr}; see also \citealp{Browman:1992da}).%
\footnote{
Note that this claim does not entail commitment to other components of the motor theory of speech perception.
}
%
On this view,
categorical perception of speech is a process which takes as input the bodily and acoustic effects of speech actions and attempts to identify which outcomes the actions are directed to bringing about, that is, which phonemes the speaker is attempting to articulate.
That isolated acoustic signals can engage this process and thereby trigger categorical perception  is merely a side-effect, albeit one with useful methodological consequences.

\subsection{slide-48}
We can think of expressions of emotion as goal-directed in the same sense that
articulations of phonemes are.
They are actions whose goal is the expression of a particular emotional episode.

This may initially strike you as implausible given that such expressions of emotion can be spontaneous, unintentional and involuntary.
But note that expressing an emotion by, say, smiling or frowning,
whether intentionally or not,
involves making coordinated movements of multiple muscles
where exactly what should move and how can depend in complex ways on  contextual factors.
That such an expression of emotion is a goal-directed action follows just from its involving motor expertise
and being coordinated around an outcome (the goal) in virtue of that outcome being represented motorically.%
\footnote{
To increase the plausibility of the conjecture under consideration, we should allow that some categorically perceived expressions of emotion are not goal-directed actions but events grounded by two or more goal-directed actions.
For ease of exposition I shall ignore this complication.
}

Recognising that some expressions of emotion are goal-directed actions in this sense makes it possible to explain
what distinguishes a genuine expression of emotion of this sort, a smile say, from something unexpressive like the exhalation of wind which might in principle resemble the smile kinematically.
Like any goal-directed actions, genuine expressions of emotion of this sort are distinguished from
their kinematically similar doppelgänger
in being directed to outcomes by virtue of the coordinating role of motor representations and processes.

\subsection{slide-49}
the wild conjecture under consideration is that the things categorical perception is supposed to categorise, the ‘expressions of emotion’, are actions of a certain type, and these are categorised by which outcomes they are directed to.

\subsection{slide-50}
Let me explain the increasingly bold commitments involved in accepting this conjecture.

First, the things categorised in categorical perception of expressions of emotion are events rather than configurations or anything static.
(Note that this is consistent the fact that static stimuli can trigger categorical perception; after all, static stimuli can also trigger motor representations of things like grasping \citep{borghi:2007_are}.)

\subsection{slide-51}
Second, these events are not mere physiological reactions (as we might intuitively take blushing to be) but things like frowning and smiling, whose performance involves motor expertise.
\footnote{
To emphasise, one consequence of this is that not everything which might intuitively be labelled as an expression of emotion is relevant to understanding what is categorised by perceptual processes.
%For example, in the right context a blush may signal emotion without requiring motor expertise.
}

\subsection{slide-52}
Third, these events are perceptually categorised  by the outcomes to which they are directed.
That is, outcomes represented motorically in performing these actions are things by which these events are categorised in categorical perception.

Should we accept the wild conjecture?
It goes well beyond the available evidence and currently lacks any reputable endorsement.
In fact, we lack direct evidence for even the first of the increasingly bold commitments just mentioned (namely, the claim that the things categorically perceived are events).
A further problem is that we know relatively little about the actions which, according to the wild conjecture, are the things categorical perception is supposed to categorise  (\citealp[p.\ 47]{scherer:2013_understanding}; see also \citealp{scherer:2007_are} and \citealp{fernandez-dols:2013_advances}).
However,
the wild conjecture is less wild than the only published responses to the problems that motivate it.%
% (which, admittedly, are wilder than an acre of snakes).%
\footnote{
See
\citet[p.\ 15]{motley:1988_facial}: ‘particular emotions simply cannot be identified from psychophysiological responses’;
and
\citet[p.\ 289]{barrett:2011_context}: ‘scientists have created an artifact’.
}
And, as I shall now explain, several considerations make the wild conjecture seem at least worth testing.

Consider again the procedure used in testing for categorical perception.
Each experiment begins with a system for categorising the stimuli (expressions).
This initial system  is either specified by the experimenters or, in some cases, by having the participants first divide stimuli into categories using verbal labels or occasionally using non-verbal decisions.
The experiment then seeks to measure whether this initial system of categories predicts patterns in discrimination.
But what determines  which category each stimulus is assigned to in  the initial system of categories?
You might guess that it is a matter of how likely people think it is that each stimulus---a particular facial configuration, say---would be associated with a particular emotion.
In fact this is wrong.
Instead,
each stimulus is categorised in the initial system according to how suitable people think such an expression would be to express a given emotion: this is true whether the stimuli are facial \citep{horstmann:2002_facial} or vocal \citep{laukka:2011_exploring} expressions of emotion (see also \citealp[pp.\ 98--9]{parkinson:2013_contextualizing}).
To repeat,
in explicitly assigning an expression to a category of emotion, people are not making a judgement about the probability of someone with that expression having that emotion:
they are making a judgement about which category of emotion the expression is most suited to expressing.
Why is this relevant to understanding what perceptual processes categorise?
The most straightforward way of interpreting the experiments on categorical perception is to suppose that they are testing whether perceptual processes categorise stimuli in the same ways as the initial system of categories does.
But we have just seen that the initial system categorises stimuli according to the emotions they would be best suited to expressing.
So on the most straightforward interpretation,
the experiments on categorical perception of expressions of emotion
are testing whether there are perceptual processes whose function is to categorise  actions of a certain type by the outcomes to which they are directed.
So the wild conjecture is needed for the most straightforward interpretation of these experiments.
This doesn't make it true but it does make it worth testing.

So far I have focussed on evidence for categorical perception from experiments using faces as stimuli.
However, there is also evidence that perceptual processes categorise vocal and facial expressions alike (\citealp{grandjean:2005_voices,laukka:2005_categorical}; see also \citealp{jaywant:2012_categorical}).
We also know that
%judgements about which emotion an observed person is expressing in a photograph can depend on the posture of the whole body and not only the face \citep{aviezer:2012_body}, and that
various contextual factors can affect how even rapidly occurring perceptual processes discriminate expressions of emotion \citep{righart:2008_rapid}.
There is even indirect evidence that categorical perception may concern whole bodies rather than just faces or voices \citep{aviezer:2008_angry,aviezer:2011_automaticity}.
In short, categorical perception of expressions of emotion plausibly resembles categorical perception of speech in being a multimodal phenomenon which concerns the whole body and is affected by several types of contextual feature.
This is consistent with the wild conjecture we are considering.
The conjecture generates the further prediction that the effects of context on categorical perception of expressions of emotion will resemble the myriad effects of context on categorical perception of speech so that `every potential cue ... is an actual cue'
 (\citealp[p.\ 11]{Liberman:1985bn}; for evidence of context effects see in categorical perception of speech, for example, \citealp{Repp:1987xo}; \citealp{Nygaard:1995po} pp.\ 72--5; \citealp{Jusczyk:1997lz}, p.\ 44).

\subsection{slide-54}
So I first asked, what evidence could bear on the perceptual hypothesis?
I answered that part of the evidence is from studies of categorical perception
of facial expressions of emotion.

Part of the evidence relevant to the perceptual hypothesis is evidence that
humans can categorically perceive expressions of emotion.

Then my next question was,
What does the evidence from studies of categorical perception
tell us about the truth or falsity of the perceptual hypothesis ?

\subsection{slide-55}
Now I think we can answer this question.
Or rather, we know that the answer depends on what the objects of categorical perception are.

If the objects of categorical perception are merely facial expresssions,
then I think the evidence does not support the claim that we see aspects of
each others’ mental lives.

But if, as I’ve conjectured, the objects of categorical perception are
actions directed to the expression of particular emotions, then I think the
evidence does provide modest support for the claim that we perceive aspects
of each others’ mental lives.

This claim will need qualifying in various ways.

\subsection{slide-56}
First, the evidence suggests that categorical perception of expressions of emotion,
like categorical perception of speech, is not tied to single modality in any very
robust way.

Second, I suggest elsewhere that categorical perception isn’t exactly like
perception as philosophers standardly understand it.
In particular, its phenomenology is to be understood in terms of
\textbf{PHENOMENAL EXPECTATIONS} rather than by analogy with perception of shapes or textures.

\subsection{slide-57}
But let me put these aside because I want to mention a second challenge.

\subsection{slide-58}
The challenge is to specify a model of anger and other mental states which captures how these
mental states appear to the perceivers. This is a hard challenge to meet because available models
arguably render mental states imperceptible.

But what is a model of the emotions?
Since emotions are mental states, here I think we can step back and think about mental states generally.
On a widely accepted view, mental states involve subjects having  attitudes toward contents (see figure).
Possible attitudes include believing, wanting, being happy that, and being angry that.
The content is what distinguishes one belief from all others, or one desire from all others.
The content is also what determines whether a belief is true or false, and whether a desire is satisfied or unsatisfied.

There are two main tasks in constructing a model of mental states.
The first task is to characterise some attitudes.
This typically involves specifying their distinctive functional and normative roles.%
\footnote{
For examples, see \citet{Bratman:1987xw} on intention or \citet[][chapter 11]{Velleman:2000fq} on belief.
}
%
The second task is to find a scheme for specifying the contents of mental states.
This typically involves one or another kind of proposition, although some  have suggested other abstract entities including map-like representations.%
\footnote{
See \citet[p.\ 163]{Braddon-Mitchell:1996ce}: ‘what is inside our heads should be thought of as more like maps than sentences.’
}

‘emotions are episodic modes of evaluative engagement with the social and practical world’
\citep[p.\ 1512]{parkinson:2008_emotions}.

FIRST LIMIT:
Determinate emotions (the particular evaluation implied by my anger in this case)
vs the determinable (anger, in this case).  Categorical perception provides evidence
only for the determinable. (Joel said in discussion that he made this point
with an analogy with CP of colour in Smith (2010)).

SECOND LIMIT:
Static emotions vs emotions unfolding (see notes below from ‘sharing a smile’).
It seems that CP doesn’t capture dynamics of emotion.

So only a very crude model of the mental would capture emotions as we
categorically perceive them.
What we need here is a minimal theory of the emotions.

\subsection{slide-60}
Motor theory of speech perception and goal ascription  ...

\subsection{slide-62}
Here are three ways of knowing: inference, testimony and perception.

My question was,
Are mental states the kinds of thing we can come to know about through
perceiving them?

\subsection{slide-63}
Focussing on emotions like joy, I have suggested that
we face two challenges (at least) in answering it.

The first challenge is to explain what kind of evidence might
bear on the question.

Here I suggested that studies of categorical perception of expressions of emotion
are relevant.
Whether they provide evidence in favour fo the view that we can know others’ mental
states through perceiving them depends on what  the objects of categorical perception
are.

If they are facial configurations, or something like this, then it seems clear that
the evidence does not support the view  that we can know others’ mental
states through perceiving them.
If anything, such evidence would support the view that when we think we are perceiving
mental states, we are actually perceiving facial expressions that we associate with
particular emotions.

But if the objects of categorical perception are actions directed to the goal of expressing
particular emotions, then studies of categorical perception do provide some evidence
for the view that we have at least phenomenal expectations concerning others’ emotions.

The second challenge concerned the limits of categorical perception.

I suggested that because categorical perception is underpinned by automatic processes
which operate over a short temporal window, the emotions which are the goals of the
actions that are the objects of categorical perception cannot be modelled as propositional
attitudes.
Instead we need a more minimal model of the mental to understand the nature of these
emotions.

In the end, we are probably not yet in a position to know whether humans ever perceptually experience others' mental states.
But thinking about categorical perception does give us a way of finally making some
progress with this question.

\subsection{sharing\_a\_smile}


\section{Sharing Smiles}

Sharing smiles and otherwise collectively expressing emotions makes available to us facets of mental states that are not available on the basis of perception.

\subsection{slide-65}
So my aim is to show that manifesting collective intentionality enables us to exploit routes to knowledge of others' minds
which would be unavailable if we were entirely passive observers.
To this end I want to focus on a one kind of interaction, sharing a smile.

Preview: I'll argue that sharing a smile provides us with a route to knowledge of facts about the ways others' emotions unfold,

where we might not otherwise be in a position to know these facts.

\subsection{slide-66}
[FIRST POINT: emotions unfold over time]
When it comes to knowledge of others’ emotions, it might initially seem hard to imagine there could be a role for joint action because mere observation provides so much.
Because there are characteristic expressions of emotion, a mere observer can know of Lily’s amusement just from her facial expression.
Expressions of emotion plausibly also enable us to identify the objects of others’ emotions.
Since the causes of expressions like Lily’s smile are often enough the objects of the emotions they express,
we can identify the objects of emotion using whatever sorts of causal reasoning enable us to work out what someone tripped over.
If you know what is causing Lily to smile you are probably in a position to know what she is amused by, that is the object of her amusement.

Emotions are not simply happiness or fear; they are things that unfold over time.
That initial jolt of fear as you hear the intruder enter your house is becomes intense as she moves closer, and then acquires an angry edge as you hear her breaking that vase your grandmother gave you.
Or imagine watching a clown falling over.  Your amusement might develop from an initial wince to growing hilarity that abruptly ends as the clown hits the floor, only to resurge as the wider comic significance of this event becomes apparent.

How could we gain insight into the fine-grained dynamics of others’ emotions?
How could we ever appreciate the unfolding of another’s grief, or the way their engagement leads to an explosion of ecstasy at the climax of a concert?
It's in answering this question that I think we may find support for the idea that collective intentionality enables us to know things about others' emtoions that we might not otherwise be in a position to know.
My thesis will be that Sharing a smile provides us with a route to knowledge of facts about the ways others' emotions unfold; this is what's special, rather than knowledge of the category of emotion (amusement vs fear) or its object.

\subsection{slide-67}
[SECOND POINT: control is a way of knowing]

ways of knowing: perception, inference, reason, testimony

ways of knowing --- mention perceiving, testimony etc;

\subsection{slide-68}
agent’s knowledge of their current and future actions is often thought to depend on control (although I guess perception is important---it's not a pure case);

\subsection{slide-69}
but control can enable us to know things about others’ actions (using reins as an example---the point of putting a toddler in reins is sometimes that they enable you to know where the toddler will be without observation or inference);
\footnote{
Issue (Joel): does control give knowledge of action independently of perception.  (I launched into AHP, which I think only ends up supporting the point.)  Perception must be relevant in this sense: to know that I’m acting, my judgement must be appropriately sensitive to perceptual confirmation.  (Someone with AHP isn’t in a position to know that she’s acting precisely because the perceptual information doesn’t influence her judgement (the comparator---the link between perceptual feedback and motor representation---is broken).  (Strictly this doesn’t show that we need perceptual information, just that knowing isn’t compatible with ignoring perceptual information if  we have it.)  OK, but how does this affect the argument?  (i) maybe I shouldn’t offer agential control as a way of knowing that is distinct from perception (although I think that, unlike simple cases of perception, the motor representation plays a role in knowing.  Further (ii) what’s perceived is action not emotion in the proposal I make; but then maybe Joel’s objection is that merely controlling isn’t sufficient for knowing---against this, I think the reins example might help (I don’t need perceptual feedback to know the location of the toddler, although I might not be able know the location if I have perceptual feedback and ignore it.)
}

\subsection{slide-70}
then point out that social interactions often involve reciprocal control.

controlling someone's actions is a way of knowing things about them.
\footnote{
Issue (Tom Smith): is reciprocal control sufficient for joint control?  No because of walking in the Tarantino sense.  This might tell us something interesting about joint control---e.g. maybe identifying differences between sharing a smile and walking in the Tarantino sense with respect to control will enable me to make the point about control as a way of knowing in sharing a smile better.
}

\subsection{slide-71}
[THIRD POINT: smiling is a goal-directed action, the goal of which is to smile that smile]

My topic is sharing a smile.  But first think about ordinary, individual actions like genuine smiles.

What distinguishes a genuine smile from a muscle spasm or the exhalation of wind?

I want to suggest that it's this: the smile is a goal-directed action where the goal is to simile that smile.

But why think of the smile as goal-directed?  Because smiling the smile requires considerable motor coordination: it’s not a matter of simple muscle contractions but more like the production of a phonetic gesture where context affects what is needed to realise the smile.

\subsection{slide-72}
Further, like grasping an object or articulating a particular phoneme, it is an action that can be realised by different bodily movements in different contexts.

This is why I put slides of two quite different but both genuine smiles.

[Objection:]

Now you might say that the smile can't be goal-directed because is isn't explicable by appeal to belief, desire and intention

This is because the genuine smile is spontaneous and not something that can be produced at will (although it could probably be inhibited, at least to some extent); after all, this is what distinguishes the genuine from the polite smile.
\footnote{
From web source: The Duchenne smile involves both voluntary and involuntary contraction from two muscles: the zygomatic major (raising the corners of the mouth) and the orbicularis oculi (raising the cheeks and producing crow's feet around the eyes). The zygomatic major can be voluntarily contracted but many people can't voluntarily contract the orbicularis oculi muscle.
}

So now we might be tempted by the view that a smile is merely caused by an emotion in the way that gasses can cause you to burp.

[Reply:]

Maybe there are smiles like this, but some genuine smiles are sustained.

And what sustains them is a process of controll

How could this be if such smiles are not consequences of beliefs, desires and intentions?

I think a reasonably natural view here is to think that part of what makes an event a smile, a goal-directed action and not just a muscle spasm caused by excess wind, is the way that motor control is involved.  Specifically, the genuine smile will involve a motor representation of the outcome, the smile, and this motor representation will lead to movements by way of planning-like motor processes.

But you don't have to buy this to agree with me.

All you have to accept is that actions like some smilings can be goal-directed and controlled even in the absence of relevant beliefs, desires and intentions.

I think smiles fall into the category of actions like graspings, reachings and gesturings which are goal-directed but do not necessarily involve intention.



So far, then, I've suggested that smiling is a goal-directed action, the goal of which is to smile that smile.

\subsection{slide-73}
I'm sorry if all of this is too simple to mention, but I want to recap just to be sure we're all together.

\subsection{slide-74}
Now imagine a situation where a single individual encounters and event (a clown’s falling) which causes amusement which causes her to smile

Note that the smile also modulates the emotion; if, for example, she supressed the smile, the quality of her amusement would change.

How could we gain insight into the fine-grained dynamics of others’ emotions?

How could we ever appreciate the unfolding of another’s grief, or the way their engagement leads to an explosion of ecstasy at the climax of a concert?

Part of the answer is obvious: by being there, with them.

[Not that this is the only possibility --- in some cases we might  be told.]

But how exactly does being there, in the same situation help?

\subsection{slide-75}
Merely being in the same situation is surely not enough.

It’s not enough that we each experience amusement, grief or ecstasy.

After all, individuals are different.  Different individuals’ feelings don’t unfold in the same way just because they are in the same situation.

It’s just here that collective intentionality is relevant.

\textbf{What is involved in sharing a smile?}

\subsection{slide-76}
Minimally, I think there have to be two kinds of connection between us for us to share a smile.

\subsection{slide-77}
First, the way your smile unfolds is shaped by how mine unfolds and conversely.

I also suppose that our smiles can be minutely coordinated with each other.

But it’s not just that our smiles are interdependent in this way ...

\subsection{slide-78}
It’s also that each of our smiles is shaping the way our amusement unfolds.

\subsection{slide-79}
So the way your amusement unfolds is being controlled by, and controlling, the way mine unfolds.

In sharing a smile, we are emotionally locked together.

\subsection{slide-80}
[*todo: remove motor stuff for this talk!  Also: don't lose sight of idea that control is a way of knowing.]

[*todo: need slide with control arrows highlighted (my emotion controls yours).]

[*Structure:
(i) I know because my emotion controls yours;
(ii) But if my emotion controls yours, how can yours be amusement at the clown's falling? because control is partial, and reciprocal;
(iii) But the mere fact of control isn't enough for knowledge; rather, control must show up in experience somehow.
  After all, for all I have said so far, we might, in sharing a smile, be  unaware that our emotions are locked together.
(iv) There must be an experience that is distinctive of sharing a smile.
(iv) Note that I don’t want to say that  someone who is sharing a smile needs to understand the situation in the way I’m describing it.
  All I'm claiming is that the fact of reciprocal control somehow affects our awareness.
(v) It may affect in our awareness insofar as we are sensitive to contingencies between our own actions' and others' actions,
  and between our actions and the causes of them.
(vi) So my position is this: the reciprocal control justifies each agent in making judgements about how the others' amusement is unfolding,
    and this justification is at least indirectly available to the agents by virtue of their having experiences characteristic of sharing a smile.
  ]

Our being emotionally locked together means that
to a significant extent I am feeling what you are feeling,
        that the way my amusement is unfolding matches they way your amusement is unfolding.
So if you know how your own amusement is unfolding and you know that we are emotionally locked together,
you can know much about how my amusement is unfolding.
So joint expressions of emotion like sharing a smile have the potential to enable us to know not just that others are amused but how their amusement is unfolding.

But the fact of reciprocal control (which means our emotions are locked together) together doesn’t all by itself mean that we can know how each other’s emotions are unfolding.
After all, for all I have said so far, we might, in sharing a smile, be  unaware that our emotions are locked together.
Now you might think this sounds implausible because its hard to imagine sharing a smile without an experience that is distinctive of sharing a smile.
And it might be natural to describe this experience as an experience of sharing.
But even if that is correct, it’s necessary to say exactly why someone who is sharing a smile is in a position to know things about how the other’s emotion is unfolding.

I don’t want to say that interaction only helps if you know that your emotions are locked together.
That is, I don’t want to say that  someone who is sharing a smile needs to understand the situation in the way I’m describing it.
But minimally the fact of reciprocal control must somehow feature in our awareness.

[*The idea in outline:
\begin{enumerate}
\item the ways our amusements unfold is locked together
\item this is in part because a single motor plan has two functions, production of your smile and prediction of my smile
\item the single motor process means that we might experience being locked together in some way (not that our emotions are locked together but that our actions are, in something like (but not exactly) we experience actions when seeing ourselves in a mirror or on CCTV (check Johannes’ discussion of this)).
\end{enumerate}
]

Here I want to offer a wild conjecture.
In joint expressions of emotion there is a single motor plan with two functions,
production and prediction.
The motor plan both produces your own smile and enables you to predict the way the other’s smile will unfold.
[*missing step about monitoring and experience.  (The Haggard idea: motor planning can give rise to experiences concerning one's own actions \citep{Haggard:2005sc}.)]
Because your plan has this dual function, your experience of the other’s (my?) smile is special.
From your point of view, it is almost as if the other is smiling your smile.%
\footnote{
Joel caricatured this idea seeing me eating fruit: ‘it’s almost as if I’m eating that fruit.’
}
This means that sharing a smile has characteristic phenomenology.

This odd phenomenological effect means that in sharing a smile we can each think of the situation almost as if there were a single smile.
And almost as if there were a single state of amusement.
(In thinking of the situation like this it is important that we have a subject-neutral conception of the amusement and an agent-neutral conception of the smile.%
\footnote{
Tom Smith asked about this.  I clarified that I wasn’t suggesting there was a state of amusement which is ours, nor that the subjects are thinking of the situation in this way.
That’s the point of the appeal to subject-neutral amusement.
It’s a partial model of the situation.
}
[*Here I think I’m shifting back from the perspective of the participants in sharing a smile to the perspective of the theorist.
Probably what I should say is, first, that a theorist can think of the situation in this way and use this to argue, second, that there is a simple, partial conception of the situation that doesn’t require understanding reciprocal control and interlocking emotions but is sufficient for each smiler to have knowledge of  the way the other’s emotion unfolds.]
So my suggestion is that in sharing a smile you experience my smile almost as if it were yours (or: you experience me almost as if I were smiling your smile),
and so you might also experience our situation almost as if it involved a single state of amusement.

It's more like we each plan a single smile.

But---to reply to the objection---these plans have a dual function.
Your plan both produces your own smile and enables you to simulate---to experience---my smile.
And likewise for my plan.
The interdependence of our smilings means that we could each think of the situation as if it were one in which a single state of amusement were responsible for our actions.

\subsection{slide-81}
I've just been arguing that manifestations of collective intentionality enable us to know things about others' minds that we might not otherwise be in a position to know.
In particular, I suggested that
sharing a smile provides us with a route to knowledge of facts about the ways others' emotions unfold.
\textbf{
Engaging in collective actions like sharing a smile (and crying together) enables us to understand others’ emotions in ways that we probably couldn’t understand them if we were mere observers and unable to engage in collective action.
The joint expression of emotion matters because it makes available a perspective from which it is almost as if your amusement is my amusement, and my grief is yours.
}
 



%--- end paste
%---------------





\bibliography{$HOME/endnote/phd_biblio}



\end{document}
