%!TEX TS-program = xelatex
%!TEX encoding = UTF-8 Unicode


\documentclass[12pt]{extarticle}
% extarticle is like article but can handle 8pt, 9pt, 10pt, 11pt, 12pt, 14pt, 17pt, and 20pt text

\def \ititle {Philosophical Psychology}

\def \isubtitle {Lecture 16}

\def \iauthor {Stephen A. Butterfill}
\def \iemail{s.butterfill@warwick.ac.uk}
\date{}

%for strikethrough
\usepackage[normalem]{ulem}

\input{$HOME/latex_imports/preamble_steve_handout}

%\bibpunct{}{}{,}{s}{}{,}  %use superscript TICS style bib
%remove hanging indent for TICS style bib
%TODO doesnt work
\setlength{\bibhang}{0em}
%\setlength{\bibsep}{0.5em}


%itemize bullet should be dash
\renewcommand{\labelitemi}{$-$}

\begin{document}

\begin{multicols*}{3}

\setlength\footnotesep{1em}


\bibliographystyle{newapa} %apalike

%\maketitle
%\tableofcontents




%---------------
%--- start paste



      
\def \ititle {16: Should You Be Instrumentally Rational?}
 
\begin{center}
 
{\Large
 
\textbf{\ititle}
 
}
 
 
 
\iemail %
 
\end{center}
 
To exhibit \emph{instrumental rationality} is to select those actions which you expect to best satisfy your preferences \citep[textbook:][]{Jeffrey:1983oe}.
 
 
‘the laws of decision theory (or any other theory of rationality) are not empirical generalisations 
about all agents. What they do is define what is meant ... by being rational’
\citep[p.~43]{Davidson:1987wc}

‘the revealed preference revolution of the 1930s (Samuelson, 1938)
  ... replaced the supposition that people are attempting to
  optimize any externally given criterion (e.g., some psychologically 
  interpretable motion of utility, perhaps to be quantified in units of pleasure and
  pain). 
  Rather, if economic agents are typically assumed to be subject to
  relatively mild consistency conditions (e.g., such as transitivity ...), 
  it can be
  shown that there will exist a set of probabilities and utilities such that each
  agent’s choices will be just “as if” that agent were maximizing expected
  utility’
  \citep{chater:2014_cognitive}.
 
What being instrumentally rational require?
‘As ordinarily understood, the prescription to maximize your expected utility
presupposes that there is some measure of expected utility that applies to you
and that your preferences are therefore obliged to maximize.
But in the context
of decision  theory, the utility and probability functions that apply to you are constructed
out of your preferences, and so your expected utility is not an independent
measure that your preferences can be obliged to maximize;
rather, your
expected utility is whatever your preferences do maximize, if they obey the
axioms.
Hence, the injunction to maximize your expected utility can at most
mean that you should have preferences that can be represented as maximizing
some measure (or measures) of expected utility, which will then apply to you by
virtue of being maximized by your preferences’
\citep[p.~149]{Velleman:2000fq}
 
 
 
\section{Motivational States}
 
‘The pattern of results accords [...] with a role for an incentive learning
process in the reinforcer devaluation effect;
not only must consumption of
the reinforcer be paired with toxicosis,
the animals must also have an
opportunity to contact the reinforcer after aversion conditioning if there is
to be a change in instrumental performance’
\citep[p.~293]{balleine:1991_instrumental}

 
‘The pattern of results accords [...] with a role for an incentive learning
process in the reinforcer devaluation effect;
not only must consumption of
the reinforcer be paired with toxicosis,
the animals must also have an
opportunity to contact the reinforcer after aversion conditioning if there is
to be a change in instrumental performance’
\citep[p.~293]{balleine:1991_instrumental}
 
‘we should search in vain among the literature for a consensus about the
psychological processes by which primary motivational states, 
such as hunger and thirst, regulate simple goal-directed [i.e. instrumental] acts’
\citep[p.~1]{dickinson:1994_motivational}
 
\subsection{Complication: Discrepant Actions}
‘The dissociation between lever pressing and magazine entries produced by
re-exposure is [...] problematic for the incentive learning account.
To
recapitulate, this explanation assumes that instrumental performance is
mediated by some “representation” of the relationship between the instrumental 
action and reinforcer that also encodes the current incentive value of
the reinforcer. The represented incentive value can only be changed, however,
after aversion conditioning by exposure to the reinforcer.
Given this account,
the question immediately arises as to why re-exposure is necessary for a change
in lever pressing but not magazine entries’
\citep[p.~293]{balleine:1991_instrumental}
 
‘A possible resolution to this discrepancy lies with the differing contingencies controlling lever pressing and magazine entry. There is evidence that simple anticipatory approach to a food source, such as magazine entry, is primarily under the control of Pavlovian as opposed to instrumental contingencies (e.g. Holland, 1979),thus raising the possibility that incentive learning is necessary for instrumental but not Pavlovian reinforcer revaluation effects. There is, in fact, independent evidence that accords with this analysis’ \citep[p.~294]{balleine:1991_instrumental}
 

\section{Dilemma}

Horn 1. Prioritise one kind of motivational state over all others. Define instrumental rationality in terms of optimally satisfying motivational states of this kind.

Horn 2. Assume that despite multiple kinds of motivational state at the level of representations and algorithms, the system as a whole will satisfy the axioms governing preferences (e.g. transitivity).

\section{Appendix: Consequences for Mindreading?}
If we have multiple, somewhat independent systems of motivational states,
how can we justify using decision theory to charaterise behaviour?

‘once we accept that there are complex and subtle non-intentional processes, such as those mediating
basic goal-approach and the adjustment to changes in motivational state, that can mimic true
intentional control in many situations, we can understand why the propensity to perceive actions as
intentional may have developed. Given that either there is nothing in the stimulus input per se to
distinguish intentional from non-intentional behaviour or that such a discrimination yields little of
consequence in most situations, it may well pay the perceiver to treat both classes of behaviour as
intentional in predicting the subsequent course of events’
\citep[p.~102]{heyes:1990_intentionality}.
 
    

\vfill






%--- end paste
%---------------

\footnotesize
\bibliography{$HOME/endnote/phd_biblio}

\end{multicols*}

\end{document}
