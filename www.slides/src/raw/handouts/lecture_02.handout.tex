%!TEX TS-program = xelatex
%!TEX encoding = UTF-8 Unicode


\documentclass[12pt]{extarticle}
% extarticle is like article but can handle 8pt, 9pt, 10pt, 11pt, 12pt, 14pt, 17pt, and 20pt text

\def \ititle {Philosophical Psychology}

\def \isubtitle {Lecture 02}

\def \iauthor {Stephen A. Butterfill}
\def \iemail{s.butterfill@warwick.ac.uk}
\date{}

%for strikethrough
\usepackage[normalem]{ulem}

\input{$HOME/Documents/submissions/preamble_steve_handout}

%\bibpunct{}{}{,}{s}{}{,}  %use superscript TICS style bib
%remove hanging indent for TICS style bib
%TODO doesnt work
\setlength{\bibhang}{0em}
%\setlength{\bibsep}{0.5em}


%itemize bullet should be dash
\renewcommand{\labelitemi}{$-$}

\begin{document}

\begin{multicols*}{3}

\setlength\footnotesep{1em}


\bibliographystyle{newapa} %apalike

%\maketitle
%\tableofcontents




%---------------
%--- start paste



\def \ititle {02: What Are Metacognitive Feelings?}

\begin{center}

{\Large

\textbf{\ititle}

}



\iemail %

\end{center}


Crude Picture of the Mind:
\begin{enumerate}
  \item epistemic
  \item motoric
  \item perceptual
\end{enumerate}

\citet[p.~302]{dokic:2012_seeds} lists some metacognitive feelings including:
\begin{itemize}
\item feelings of knowing or not knowing \citep{koriat:2000_feeling}
\item tip-of-the-tongue experiences (Brown 2000; Schwarz 2002)
\item feelings of certainty or uncertainty (Smith et al. 2003)
\item feelings of confidence (Winman and Juslin 2005)
\item feelings of ease of learning (Koriat 1997)
\item feelings of competence (Bjork and Bjork 1992)
\item feelings of ‘déjà vu’ (Brown 2003)
\item feelings of rationality or irrationality (James 1879)
\item feelings of rightness (Thomson 2008)
\end{itemize}

\section{Familiarity}

The feeling of familiarity is not a consequence of how familiar things 
actually are; instead it may be a consequence of 
the degree of fluency with which unconscious processes can identify 
perceived items \citep{Whittlesea:1993xk,Whittlesea:1998qj}.

Learning a grammar can also generate feelings of familiarity \citep{scott:2008_familiarity}.

Subjects are also not doomed to treat feelings of familiarity as being
about actual familiarity:
instead subjects can use feeling of 
familiarity in deciding whether a stimulus is from that grammar 
\citep{Wan:2008_familiarity}.

\section{Is there a metacognitive feeling of surprise?}

‘the intensity of felt surprise is not only influenced by the unexpectedness of the surprising event, but also by the degree of the event’s interference with ongoing mental activity, [...]
the effect of unexpectedness on surprise is [...] partly mediated by mental interference’
\citep[p.~271]{reisenzein2000subjective}.
That is, the feeling of surprise is a sensational consequence of mental interference.

\citet[p.~79]{foster:2015_whya} appear to offer a conflicting view:
‘the MEB theory of surprise posits that: Experienced surprise is a metacognitive assessment
of the cognitive work carried out to explain an outcome. Very surprising events are those
that are difficult to explain, while less surprising events are those which are easier to
explain.’
However, \citet{foster:2015_whya} is about reactions to reading about something unexpected, whereas
\citet{reisenzein2000subjective} measures how people experience unexpected events (changes to
stimuli while solving a problem).

\section{Is there a metacognitive feeling of agency?}

Feelings of agency seem to arise
from a number of cues including   comparison between outcomes represented motorically and outcomes detected sensorily and  the fluency of an action selection process (that is, the ease or difficulty involved in 
selecting one among several possible actions to perform motorically).
The latter can be manipulated by,
for example, providing helpful or misleading cues to action
\citep{wenke:2010_subliminal,sidarus:2013_priming,sidarus:2017_how}.

‘the SoA [sense of agency] may provide an important experiential marker, both for alerting to
the need for corrective action, and for guiding learning’
\citep[p.~11]{sidarus:2017_how}


\section{What are metacognitive feelings?}
Are they aspects of the overall phenomenal character of experiences
which their subjects take to be informative about things that are only
distantly related (if at all) to the things that those experiences
intentionally relate the subject to?

Can metacognitive feelings  be thought of as \emph{sensations} in approximately Reid’s sense? I.e. they are
monadic properties of events, specifically perceptual experiences, which are individuated by
their normal causes and which alter the overall phenomenal character of those experiences in
ways not determined by the experiences’ contents (so two perceptual experiences can have the
same content but distinct sensational properties).

Like sensations, metacognitive feelings  can lead to beliefs via learnt associations (compare \citealp[Essay~II, Chap.~16, p.~228]{Reid:1785cj};
\citealp[Chap.~VI sect.~III, pp.~164–5]{Reid:1785nz}).


Dokic’s ‘Water Diviner’ model:
‘noetic [metacognitive] feelings ... are first-order bodily experiences, namely non-sensory affective experiences about bodily states, which given our brain architecture co-vary with first-order epistemic states, in such a way that they can be recruited, through some kind of learning or association process, to represent conditions hinging on relevant epistemic properties of one’s own mind’
\citep[p.~317]{dokic:2012_seeds}.

If this is right, metacognitive feelings do not involve representation. As \citet[p.~310]{dokic:2012_seeds}suggests, ‘the causal antecedents of noetic [metacognitive] feelings
can be said to be metacognitive insofar as they involve implicit monitoring mechanisms that
are sensitive to non-intentional properties of first-order cognitive processes.’


\section{Why do humans have metacognitive feelings?}

‘metacognitive feelings ... allow a transition from the implicit-automatic mode to the
explicit-controlled mode of operation.’
\citep[p.~150]{koriat:2000_feeling}




%--- end paste
%---------------

\footnotesize
\bibliography{$HOME/endnote/phd_biblio}

\end{multicols*}

\end{document}
