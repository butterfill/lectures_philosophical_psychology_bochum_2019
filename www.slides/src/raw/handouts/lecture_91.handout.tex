%!TEX TS-program = xelatex
%!TEX encoding = UTF-8 Unicode


\documentclass[12pt]{extarticle}
% extarticle is like article but can handle 8pt, 9pt, 10pt, 11pt, 12pt, 14pt, 17pt, and 20pt text

\def \ititle {Origins of Mind}

\def \isubtitle {Lecture 01}

\def \iauthor {Stephen A. Butterfill}
\def \iemail{s.butterfill@warwick.ac.uk}
\date{}

%for strikethrough
\usepackage[normalem]{ulem}

\input{$HOME/latex_imports/preamble_steve_handout}

%\bibpunct{}{}{,}{s}{}{,}  %use superscript TICS style bib
%remove hanging indent for TICS style bib
%TODO doesnt work
\setlength{\bibhang}{0em}
%\setlength{\bibsep}{0.5em}


%itemize bullet should be dash
\renewcommand{\labelitemi}{$-$}

\begin{document}

\begin{multicols*}{3}

\setlength\footnotesep{1em}


\bibliographystyle{newapa} %apalike

%\maketitle
%\tableofcontents




%---------------
%--- start paste



\def \ititle {Lecture 01: Intention and Motor Representation in Purposive Action}

\begin{center}

{\Large

\textbf{\ititle}

}



\iemail %

\end{center}




\section{Motor Representation}

‘a given motor act may change both as a function of what motor act will follow it—a sign of
planning’ \citep[p.~294]{cohen:2004_wherea}.


Markers of motor representation
\begin{enumerate}
\item are unaffected by variations in kinematic features but not goals
  \citep[e.g.][]{cattaneo:2010_state-dependent,umilta:2008pliers,rochat:2010_responses}
\item are affected by variations in goals but not kinematic features
  \citep[e.g.][]{Fogassi:2005nf,bonini:2010_ventral,Umilta:2001zr,villiger:2010_activity,koch:2010_resonance}
\end{enumerate}
So:
\begin{enumerate}[resume]
\item carry information about goals (from 1,2)
\end{enumerate}
Also
\begin{enumerate}[resume]
\item Information about outcomes guides planning-like processes
  \citep[consider][]{grafton:2007_evidence,jeannerod:1998nbo,wolpert:1995internal, miall:1996_forward,mason:2001_hand,santello:2002_patterns}.
\end{enumerate}






\section{Motor Representations Aren’t Intentions}

Imagining seeing an object move and actually seeing an object move have similarities in
characteristic performance profile
(\citealp{kosslyn:1978_measuring}; \citealp[p.\ 99ff]{kosslyn:1994_image}; \citealp{kosslyn:1978_visual})

The way imagining performing an action unfolds in time is
similar in some respects to the way actually performing an action of the same type would unfold
\citep{decety:1989_timing, Jeannerod:1994oz, parsons:1994_temporal,
frak:2001_orientation}

Judging the laterality of a hand vs of a letter.
For ordinary subjects, the tasks differ: they are less accurate
when the hand's position is biomechanically awkward.
But \citet{Fiori:2012fk} show that the tasks do not so differ for subjects suffering Amyotrophic
Lateral Sclerosis (ALS), which impairs motor representation \citep{parsons:1998_cerebrally}.

\begin{enumerate}
\item Only representations with a common format can be inferentially integrated.
\item Any two intentions can be inferentially integrated in practical reasoning.
\item My intention that I visit the ZiF is a propositional attitude.
\end{enumerate}
Therefore:
\begin{enumerate}[resume]
\item  All intentions are propositional attitudes
\end{enumerate}
But:
\begin{enumerate}[resume]
\item No motor representations are propositional attitudes.
\end{enumerate}
So:
\begin{enumerate}[resume]
\item No motor representations are intentions.
\end{enumerate}



\section{The Interface Problem}

The interface problem: explain how intentions and motor representations, with their distinct
representational formats, are related in such a way that, in at least some cases, the outcomes they
specify non-accidentally match.

‘both mundane cases of action slips and pathological conditions, such as apraxia or anarchic hand
syndrome (AHS), illustrate the existence of an interface problem’
\citep[p.~7]{mylopoulos:2016_intentions}.

Two collections of outcomes, A and B, \emph{match} in a particular context just if, in that context,
either the occurrence of the A-outcomes would normally constitute or cause, at least partially, the
occurrence of the B-outcomes or vice versa. 





\vfill


%--- end paste
%---------------

\footnotesize
\bibliography{$HOME/endnote/phd_biblio}

\end{multicols*}

\end{document}
