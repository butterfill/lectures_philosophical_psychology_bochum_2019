%!TEX TS-program = xelatex
%!TEX encoding = UTF-8 Unicode

\documentclass[12pt]{extarticle}
% extarticle is like article but can handle 8pt, 9pt, 10pt, 11pt, 12pt, 14pt, 17pt, and 20pt text

\def \ititle {Philosophical Psychology}

\def \isubtitle {Lecture 08}

\def \iauthor {Stephen A. Butterfill}
\def \iemail{s.butterfill@warwick.ac.uk}
\date{}

%for strikethrough
\usepackage[normalem]{ulem}

\input{$HOME/latex_imports/preamble_steve_handout}

%\bibpunct{}{}{,}{s}{}{,}  %use superscript TICS style bib
%remove hanging indent for TICS style bib
%TODO doesnt work
\setlength{\bibhang}{0em}
%\setlength{\bibsep}{0.5em}


%itemize bullet should be dash
\renewcommand{\labelitemi}{$-$}

\begin{document}

\begin{multicols*}{3}

\setlength\footnotesep{1em}


\bibliographystyle{newapa} %apalike

%\maketitle
%\tableofcontents




%---------------
%--- start paste





\def \ititle {Lecture 08: Mindreading / Metacognitive Feelings}

\begin{center}

{\Large

\textbf{\ititle}

}



\iemail %

\end{center}



\emph{Dual Process Theory} (core part):
Two (or more) processes for tracking Xs are distinct:
the conditions which influence whether they occur,
and which outputs they generate,
do not completely overlap.




\emph{Metacognitive feelings}
are aspects of the overall phenomenal character of experiences
which their subjects take to be informative about things that are only
distantly related (if at all) to the things that those experiences
intentionally relate the subject to.

Metacognitive feelings can be thought of as \emph{sensations} in approximately Reid’s sense: they are
monadic properties of events, specifically perceptual experiences, which are individuated by
their normal causes and which alter the overall phenomenal character of those experiences in
ways not determined by the experiences’ contents (so two perceptual experiences can have the
same content but distinct sensational properties).

Like sensations, metacognitive feelings  can lead to beliefs via learnt associations (compare \citealp[Essay~II, Chap.~16, p.~228]{Reid:1785cj};
\citealp[Chap.~VI sect.~III, pp.~164–5]{Reid:1785nz}).

‘the intensity of felt surprise is not only influenced by the unexpectedness of the surprising event, but also by the degree of the event’s interference with ongoing mental activity, [...]
the effect of unexpectedness on surprise is [...] partly mediated by mental interference’
\citep[p.~271]{reisenzein2000subjective}

‘the SoA [sense of agency] may provide an important experiential marker, both for alerting to
the need for corrective action, and for guiding learning’
\citep[p.~11]{sidarus:2017_how}

‘metacognitive feelings ... allow a transition from the implicit-automatic mode to the
explicit-controlled mode of operation.’
\citep[p.~150]{koriat:2000_feeling}

\emph{The Assumption of Representational Connections}: the transition involves operations on the
contents of core knowledge states, which transform them into (components of) the contents of
knowledge states.

Many proposals rely on this Assumption, including:
(i) Spelke’s suggestion that mature understanding of objects derives from core knowledge by
virtue of core knowledge representations being assembled (\citeyear{Spelke:2000nf}); (ii) claims
by Leslie and others that modules provide conceptual identifications of their inputs
\citep{Leslie:1988ct}; (iii) Karmiloff-Smith’s representational re-description
(\citeyear{Karmiloff-Smith:1992lv}); and (iv) Mandler’s claim that ‘the earliest conceptual
functioning consists of a redescription of perceptual structure’ (\citeyear{Mandler:1992vn}).

\emph{Conjecture}
Only metacognitive feelings
(and behaviours and other intentional isolators)
connect early-developing processes for tracking objects, causes, actions and minds
to the epistemic.




%--- end paste
%---------------

\footnotesize
\bibliography{$HOME/endnote/phd_biblio}

\end{multicols*}

\end{document}
