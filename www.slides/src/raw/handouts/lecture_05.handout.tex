%!TEX TS-program = xelatex
%!TEX encoding = UTF-8 Unicode

\documentclass[12pt]{extarticle}
% extarticle is like article but can handle 8pt, 9pt, 10pt, 11pt, 12pt, 14pt, 17pt, and 20pt text

\def \ititle {Philosophical Psychology}

\def \isubtitle {Lecture 05}

\def \iauthor {Stephen A. Butterfill}
\def \iemail{s.butterfill@warwick.ac.uk}
\date{}

%for strikethrough
\usepackage[normalem]{ulem}

\input{$HOME/latex_imports/preamble_steve_handout}

%\bibpunct{}{}{,}{s}{}{,}  %use superscript TICS style bib
%remove hanging indent for TICS style bib
%TODO doesnt work
\setlength{\bibhang}{0em}
%\setlength{\bibsep}{0.5em}


%itemize bullet should be dash
\renewcommand{\labelitemi}{$-$}

\begin{document}

\begin{multicols*}{3}

\setlength\footnotesep{1em}


\bibliographystyle{newapa} %apalike

%\maketitle
%\tableofcontents




%---------------
%--- start paste




      
\def \ititle {05: What Is the Mark That Distinguishes Actions?}
 
\begin{center}
 
{\Large
 
\textbf{\ititle}
 
}
 
 
 
\iemail %
 
\end{center}
 
 
 
 
\section{Intentions and Goals}
 
What is the relation between a purposive joint action and the outcome or outcomes to which it is directed?
One way of answering this question appeals to intentions.
On any standard view, an intention
represents an outcome, causes an action, and does so in a way that would normally facilitate
the outcome’s occurrence. 
 
 
\section{Motor Representation}
 
Markers of motor representation 
\begin{enumerate}
\item are unaffected by variations in kinematic features but not goals 
  \citep[e.g.][]{cattaneo:2010_state-dependent,umilta:2008pliers,cattaneo:2009_representation,rochat:2010_responses}
\item are affected by variations in goals but not kinematic features
  \citep[e.g.][]{Fogassi:2005nf,bonini:2010_ventral,cattaneo:2007_impairment,Umilta:2001zr,villiger:2010_activity,koch:2010_resonance}
\end{enumerate}
So:
\begin{enumerate}[resume]
\item carry information about goals (from 1,2)
\end{enumerate}
Also
\begin{enumerate}[resume]
\item Information about outcomes guides planning-like processes
  \citep[consider][]{grafton:2007_evidence,jeannerod:1998nbo,wolpert:1995internal, miall:1996_forward,arbib:1985_coordinated,mason:2001_hand,santello:2002_patterns}.
\end{enumerate}
 
‘a given motor act may change both as a function of what motor act will follow it—a sign of
planning—and as a function of what motor act preceded it—a sign
of memory’ \citep[p.~294]{cohen:2004_wherea}.
 
 
 
\section{Motor Representations Ground the Directedness of Actions to Goals}

Like intentions, motor representations (i) represent outcomes, (ii) coordinate actions and (iii) do so in ways that would normally facilitate the occurrence of the represented outcomes.

\section{Motor Representations Aren’t Intentions}

Imagining seeing an object move and actually seeing an object move have similarities in
characteristic performance profile
(\citealp{kosslyn:1978_measuring}; \citealp[p.\ 99ff]{kosslyn:1994_image}; \citealp{kosslyn:1978_visual})

The way imagining performing an action unfolds in time is
similar in some respects to the way actually performing an action of the same type would unfold
\citep{decety:1989_timing, decety:1996_imagined, Jeannerod:1994oz, parsons:1994_temporal,
frak:2001_orientation}

Judging the laterality of a hand vs of a letter.
For ordinary subjects, the tasks differ: they are less accurate
when the hand's position is biomechanically awkward.
But \citet{Fiori:2012fk} show that the tasks do not so differ for subjects suffering Amyotrophic
Lateral Sclerosis (ALS), which impairs motor representation \citep{parsons:1998_cerebrally}.

\begin{enumerate}
\item Only representations with a common format can be inferentially integrated.
\item Any two intentions can be inferentially integrated in practical reasoning.
\item My intention that I visit the ZiF is a propositional attitude.
\end{enumerate}
Therefore:
\begin{enumerate}[resume]
\item  All intentions are propositional attitudes
\end{enumerate}
But:
\begin{enumerate}[resume]
\item No motor representations are propositional attitudes.
\end{enumerate}
So:
\begin{enumerate}[resume]
\item No motor representations are intentions.
\end{enumerate}

    


\footnotesize
\setlength{\bibsep}{6pt}
\bibliography{$HOME/endnote/phd_biblio}

\end{multicols*}

\end{document}
