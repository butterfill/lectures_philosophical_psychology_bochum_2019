%!TEX TS-program = xelatex
%!TEX encoding = UTF-8 Unicode


\documentclass[12pt]{extarticle}
% extarticle is like article but can handle 8pt, 9pt, 10pt, 11pt, 12pt, 14pt, 17pt, and 20pt text

\def \ititle {Origins of Mind}

\def \isubtitle {Lecture 01}

\def \iauthor {Stephen A. Butterfill}
\def \iemail{s.butterfill@warwick.ac.uk}
\date{}

%for strikethrough
\usepackage[normalem]{ulem}

\input{$HOME/latex_imports/preamble_steve_handout}

%\bibpunct{}{}{,}{s}{}{,}  %use superscript TICS style bib
%remove hanging indent for TICS style bib
%TODO doesnt work
\setlength{\bibhang}{0em}
%\setlength{\bibsep}{0.5em}


%itemize bullet should be dash
\renewcommand{\labelitemi}{$-$}

\begin{document}

\begin{multicols*}{3}

\setlength\footnotesep{1em}


\bibliographystyle{newapa} %apalike

%\maketitle
%\tableofcontents




%---------------
%--- start paste


      
\def \ititle {Lecture 03: \\ Experience of Action}
 
\begin{center}
 
{\Large
 
\textbf{\ititle}
 
}
 
 
 
\iemail %
 
\end{center}
 
 
 
\section{The Interface Problem (Recap)}
 
How does it come about that intentions and motor representations, with their distinct
representational formats, are related in such a way that, in at least some cases, the outcomes they
specify non-accidentally match?

 
 
\section{The New Interface Problem}

\begin{enumerate} 
\item In action observation, motor representations of outcomes
sometimes facilitate the identification of goals in thought.

\item So where motor representations influence a thought about an action being directed to a particular outcome, there is normally a motor representation of this outcome, or of a matching outcome.

\item But how could motor representations have content-respecting influences on thoughts given their inferential isolation?
\end{enumerate}
 
 
 
 
\section{Mylopoulos and Pacherie’s Proposal}
 
‘As defined by Tutiya et al., an executable concept of a type of movement is a
representation, that could guide the formation of a volition, itself the proximal cause of
a corresponding movement. Possession of an executable concept of a type of movement thus
implies a capacity to form volitions that cause the production of movements that are
instances of that type’
(\citealp[p.~7]{pacherie:2011_nonconceptual}; \citealp{mylopoulos:2016_intentions}).
 
Other proposals include \citet{burnston:2017_interface} and \citet{shepherd:2018_skilled}.
 
 
\section{A Question about Experiences of (Speech) Actions}

Indirect Hypothesis:  experiences revelatory of action are all experiences of bodily configurations, of joint displacements and of effects characteristic of particular actions.  Some such experiences are influenced by motor representations in ways that reliably improve veridicality.  And such experiences can provide reasons for judgements about the goals of actions providing that the subject knows, or is entitled to rely on, certain facts about which bodily configurations, joint displacements and  sensory effects are characteristic of which actions.

Direct Hypothesis some experiences revelatory of action are experiences of actions as directed to particular outcomes.  In observing action we experience not only bodily configurations, joint displacements, sounds and the rest but also goal-directed actions.  Further, such experiences stand to motor representations somewhat as perceptual experiences stand to perceptual representations.  These experiences provide reasons for judgements in something like the way that, on some views, perceptual experience of a physical object might provide a reason for a judgement about that object.
 
 
\section{Action Experience}
 
The \emph{Action Index Conjecture}:
motor representations of outcomes structure 
experiences, imaginings and (prospective) memories
in ways which provide opportunities for attention to actions directed to those outcomes.
Forming intentions concerning an outcome can influence attention to the action,
which can influence the persistence of a motor representation of the outcome.
    


\vfill


%--- end paste
%---------------

\footnotesize
\bibliography{$HOME/endnote/phd_biblio}

\end{multicols*}

\end{document}
