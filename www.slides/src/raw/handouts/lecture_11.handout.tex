%!TEX TS-program = xelatex
%!TEX encoding = UTF-8 Unicode


\documentclass[12pt]{extarticle}
% extarticle is like article but can handle 8pt, 9pt, 10pt, 11pt, 12pt, 14pt, 17pt, and 20pt text

\def \ititle {Philosophical Psychology}

\def \isubtitle {Lecture 11}

\def \iauthor {Stephen A. Butterfill}
\def \iemail{s.butterfill@warwick.ac.uk}
\date{}

%for strikethrough
\usepackage[normalem]{ulem}

\input{$HOME/latex_imports/preamble_steve_handout}

%\bibpunct{}{}{,}{s}{}{,}  %use superscript TICS style bib
%remove hanging indent for TICS style bib
%TODO doesnt work
\setlength{\bibhang}{0em}
%\setlength{\bibsep}{0.5em}


%itemize bullet should be dash
\renewcommand{\labelitemi}{$-$}

\begin{document}

\begin{multicols*}{3}

\setlength\footnotesep{1em}


\bibliographystyle{newapa} %apalike

%\maketitle
%\tableofcontents




%---------------
%--- start paste



\def \ititle {11: The Motor Theory of Goal Tracking}

\begin{center}

{\Large

\textbf{\ititle}

}



\iemail %

\end{center}

 
 
\section{The Motor Theory of Goal Tracking}
 

According to the  Motor Theory, infants’ (and adults’) pure goal-tracking sometimes depends on the double life of motor processes  \citep[see][for details]{sinigaglia:2015_puzzle}.
 
More carefully the \emph{Motor Theory of Goal Tracking} states that:
\begin{enumerate}
\item in action observation, possible outcomes of observed actions are represented motorically;
\item these representations trigger motor processes much as if the observer were performing actions directed to the outcomes;
\item such processes generates predictions;
\item a triggering representation is weakened if the predictions it generates fail.
\end{enumerate}
The result is that, often enough, the only only outcomes to which the observed action is a means
are represented strongly.
 
 

\vfill


%--- end paste
%---------------

\footnotesize
\bibliography{$HOME/endnote/phd_biblio}

\end{multicols*}

\end{document}
