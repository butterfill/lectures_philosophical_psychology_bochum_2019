%!TEX TS-program = xelatex
%!TEX encoding = UTF-8 Unicode

\documentclass[12pt]{extarticle}
% extarticle is like article but can handle 8pt, 9pt, 10pt, 11pt, 12pt, 14pt, 17pt, and 20pt text

\def \ititle {Philosophical Psychology}

\def \isubtitle {Lecture 04}

\def \iauthor {Stephen A. Butterfill}
\def \iemail{s.butterfill@warwick.ac.uk}
\date{}

%for strikethrough
\usepackage[normalem]{ulem}

\input{$HOME/Documents/submissions/preamble_steve_handout}

%\bibpunct{}{}{,}{s}{}{,}  %use superscript TICS style bib
%remove hanging indent for TICS style bib
%TODO doesnt work
\setlength{\bibhang}{0em}
%\setlength{\bibsep}{0.5em}


%itemize bullet should be dash
\renewcommand{\labelitemi}{$-$}

\begin{document}

\begin{multicols*}{3}

\setlength\footnotesep{1em}


\bibliographystyle{newapa} %apalike

%\maketitle
%\tableofcontents




%---------------
%--- start paste




\def \ititle {04: Decision Theory and Habitual Processes}

\begin{center}

{\Large

\textbf{\ititle}

}



\iemail %

\end{center}




 
 
\section{Decision Theory}
 
 
To exhibit \emph{instrumental rationality} is to select those actions which you expect to best satisfy your preferences \citep[textbook:][]{Jeffrey:1983oe}.



\section{Game Theory}
A game is ‘any interaction between agents that is governed by a set of rules 
specifying the possible moves for each participant and a set of outcomes for each possible combination of moves’
\citep[p.~3]{hargreaves:2004_game}
 
‘A game is a description of strategic interaction that includes the constraints on the actions that the players can take and the players’ interests, but does not specify the actions that the players do take’
\citep[p.~2]{osborne:1994_game}.
 
‘All situations in which at least one agent can only act to maximize his utility through anticipating (either consciously, or just implicitly in his behavior) the responses to his actions by one or more other agents is called a game’
\citep{ross:2018_game}.


When two or more agents interact,
so that which outcome one agent’s choice brings about depends on how another chooses,
how do their preferences guide their choices?
 
‘we wish to find the mathematically complete principles which define “rational behavior” for the participants in a social economy, and to derive from them the general characteristics of that behavior’
\citep[p.~31]{neumann:1953_theory}
 
Decision Theory is about how individuals decide which of several available actions to perform
\citep[textbook:][]{Jeffrey:1983oe}.
Game Theory is a development which focusses on how interacting individuals 
select actions when 
which outcome one individuals’s action brings about depends on how another acts.
 
 
\subsection{Common Knowledge of Rationality}

Keynesian Beauty Contest

‘It is not a case of choosing those [faces] that, to the best of one's judgment, are really the prettiest, nor even those that average opinion genuinely thinks the prettiest. We have reached the third degree where we devote our intelligences to anticipating what average opinion expects the average opinion to be. And there are some, I believe, who practice the fourth, fifth and higher degrees’ \citep{keynes:1936_general}.

The \emph{Harsanyi-Aumann Doctrine}:
‘in every finite game, the prior beliefs of every rational player who knows the rules of the game are the same’ \citep{hargreaves:2004_game}.

Consequence: ‘rational players with common knowledge of rationality will not be able to agree to disagree on the likelihood of any action in the game.’
 
 
\section{Descision Theory Is Agnostic about Processes}
 
On explanation: ‘Many events and outcomes prompt us to ask: Why did that happen?  [...] For example, cutthroat competition in business is the result of the rivals being  trapped in a prisoners’ dilemma’ 
\citep[p.~36]{dixit:2014_games}.
 
\begin{enumerate}
\item Applications of game theory range from interactions between microbial populations to  interactions between countries.
\item The explanations are of the same type in every case.
\item The underlying processes probably differ.
\item Therefore, game theory is agnostic about processes.
\end{enumerate}
 
 
 
\section{Processes: Habitual vs Instrumental}
 
What kinds of processes in individual animals guide actions?
 
Habitual processes are characterised by \emph{Thorndyke’s Law of Effect}:
‘The presenta­tion of an effective [=rewarding] outcome following an action [...] rein­forces 
a connection between the stimuli present when the action is per­formed and the action itself  
so that subsequent presentations of these stimuli elicit the [...] action as a response’ 
\citep[p.48]{Dickinson:1994sm}.

Instrumental processes are characterised by a different principle: 
performing actions enables agents to form expectations about their outcomes;
and the occurrence of outcomes enables agents to learn about how valuable these outcomes are.
Whether an agent performs an action depends on (a) her expectation about its outcome; and (b) her preferences concerning that outcome. 

 
 
\section{A Puzzle about Action}
 
To \emph{devalue} something is to eliminate or reverse an agent’s preference for it.
For example, a food that is novel to an agent can be devalued by making the agent ill shortly after she has consumed it and then confronting her with the food again.
This will cause the agent to become averse to the food.

‘the laboratory rat fits the teleological [instrumental] model; performance of this particular instrumental behaviour really does seem to be controlled by knowledge about the relation between the action and the goal’
\citep[p.~72]{Dickinson:1985qp}
 
‘we did not conclude that all such responding was of this form.
Indeed, we observed some residual
responding during the post-re-valuation test that appeared to be impervious to outcome devaluation
and therefore autonomous of the current incentive value,
and we speculated that this responding was
habitual
and established by a process akin to the stimulus-response (S-R)/reinforcement mechanism
embodied in Thorndike’s classic Law of Effect (Thorndike, 1911).
\citep[p.~179]{dickinson:2016_instrumental}
 
The puzzle:
\begin{enumerate}
  \item If the action is habitual, 
  why is it modulated by devaulation?
  
  \item If the action is instrumental, 
  why does it still occur (albeit less frequently) after devaluation?
\end{enumerate}
 
 
 
\section{A Dual-Process Theory of Action}
 
Some actions are ‘controlled by two dissociable processes: a goal-directed [instrumental] and an habitual process’
\citep{Dickinson:1985qp,dickinson:2016_instrumental}.

Dickinson’s dual-process theory is supported by (a) confirmation of its predictions concerning observed behaviour; and 
(b) neurophysiological discoveries.

On neurophysiological discoveries: ‘goal-directed and habitual control have been doubly dissociated in two brain regions.
In the PFC, lesions of the prelimbic and infralimbic areas disrupt goal-directed and habitual
behavior 
These dissociations suggest that different neural
circuits mediate the two forms of control’
\citep[p~184]{dickinson:2016_instrumental}

  
 
\section{Stress}
 
‘instrumental behavior itself involves two systems, the goal-directed and the habitual’
\citep[p.~12]{dickinson:2018_actions}
 
When stressed,
your preferences matter less:
habits dominate \citep{schwabe:2010_socially}.


\section{Training Effects}
Whether you learn about the effects of an action
can influence
whether that action becomes dominated by instrumental or habitual processes \citep{klossek:2011_choice}.

‘We argued that the variation in the development of behavioral autonomy arose from the different contingency experienced of the two groups. Once responding at a high and constant rate in the single-action condition after extended training, agents no longer experience the full causal contingency, speci cally episodes in which they do not respond and do not receive the outcome. As a result, the action-outcome causal representation necessary for goal-directed action is not maintained’
\citep[p.~181]{dickinson:2016_instrumental}.
 
 
 
    

 
 
 
\section{Construals of Decision Theory}
 
My proposal:
\begin{quote}
Decision Theory (like Game Theory) specifies a model of action.
Models can be construed in several different ways.
Decision Theory says nothing about how the model should be construed.
\end{quote}
Alternatives exist.  For instance, \citet{binmore:1994_playing} claims the axioms
of game theory are tautologies; on his story, the games are the models.


\subsection{Models}
‘Theories, as they are usually understood by philosophers, make claims about the
world [...]
Models, in my sense, do not themselves say anything about the
world.
Models are structures that can be used by scientists to say various
different things about the world,
by means of commentaries that accompany
models but are distinct from them’
\citep[p.~4]{godfrey-smith:2005_folk}.
 
‘Two scientists can use the same model to help with the same target system while having quite different views of how the model might be representing the target system. I will describe this situation by saying that the two scientists have different construals of the model’ 
\citep[p.~4]{godfrey-smith:2005_folk}
 
‘one scientist might [construe] some model simply as an input-output device, as a
predictive tool.
 Another might [construe] the same model as a faithful map of the
inner workings of the target system’
\citep[p.~4]{godfrey-smith:2005_folk}
 
‘it is ... possible to have facility with the model, and have a sense of which target systems are appropriate for it, while not having much of a construal at all’ 
\citep[p.~5]{godfrey-smith:2005_folk}.

\subsection{Link to Mindreading}
‘Basic facility with the folk-psychological model does not require using a
particular construal of it. Many construals are possible.
And it is also possible to have facility with the model, and have a sense of which target systems are appropriate for it, while not having much of a construal at all’ 
\citep[p.~5]{godfrey-smith:2005_folk}.
 
‘we should think of meanings and beliefs as interrelated constructs of a single theory just 
as we already view subjective values and probabilities as interrelated constructs of decision theory’
\citep[p.~146]{Davidson:1974gh}


\subsection{Possible construals}
‘we should think of meanings and beliefs as interrelated constructs of a single theory just 
as we already view subjective values and probabilities as interrelated constructs of decision theory’
\citep[p.~146]{Davidson:1974gh}

Is the revealed preference theory construal trivial?
‘Completeness applies to preference as choice, while transitivity applies to preference 
as a set of judgments of well-being. Convincing arguments for the axioms taken together 
cannot be assembled on either definition.’
\citep[p.~374]{mandler:2001_difficult}

\subsection{Rationality}
‘the laws of decision theory (or any other theory of rationality) are not empirical generalisations 
about all agents. What they do is define what is meant ... by being rational’
\citep[p.~43]{Davidson:1987wc}

\subsection{Normativity}
On the normative construal, what does decision theory demand of rational agents?
‘As ordinarily understood, the prescription to maximize your expected utility
presupposes that there is some measure of expected utility that applies to you
and that your preferences are therefore obliged to maximize.
But in the context
of decision  theory, the utility and probability functions that apply to you are constructed
out of your preferences, and so your expected utility is not an independent
measure that your preferences can be obliged to maximize;
rather, your
expected utility is whatever your preferences do maximize, if they obey the
axioms.
Hence, the injunction to maximize your expected utility can at most
mean that you should have preferences that can be represented as maximizing
some measure (or measures) of expected utility, which will then apply to you by
virtue of being maximized by your preferences’
\citep[p.~149]{Velleman:2000fq}
 
 

\section{An Interface Problem}

‘Our version of dual-system theory assumes that the outputs of the goal-directed and habitual systems summate in generating behavior but fails to offer commensurate psychologies for the two systems that would allow for such summation. 
We appeal to an intentional psychology involving the process of practical inference to explain goal-directed action, 
 whereas habitual responding is attributed to a mechanistic psychology in which the process of excitation (and inhibition) operates through associative connections’ 
\citep[p.~19]{dickinson:2018_actions}
 
‘Dickinson (2012) has suggested that this disjunction might be resolved by an associative account of practical inference within the processing architecture of an associative-cybernetic model’
\citep[p.~19]{dickinson:2018_actions}.
 

    
%--- end paste
%---------------

\footnotesize
\bibliography{$HOME/endnote/phd_biblio}

\end{multicols*}

\end{document}
